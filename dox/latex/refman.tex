\documentclass{book}
\usepackage[a4paper,top=2.5cm,bottom=2.5cm,left=2.5cm,right=2.5cm]{geometry}
\usepackage{makeidx}
\usepackage{natbib}
\usepackage{graphicx}
\usepackage{multicol}
\usepackage{float}
\usepackage{listings}
\usepackage{color}
\usepackage{ifthen}
\usepackage[table]{xcolor}
\usepackage{textcomp}
\usepackage{alltt}
\usepackage{ifpdf}
\ifpdf
\usepackage[pdftex,
            pagebackref=true,
            colorlinks=true,
            linkcolor=blue,
            unicode
           ]{hyperref}
\else
\usepackage[ps2pdf,
            pagebackref=true,
            colorlinks=true,
            linkcolor=blue,
            unicode
           ]{hyperref}
\usepackage{pspicture}
\fi
\usepackage[utf8]{inputenc}
\usepackage{polski}
\usepackage[T1]{fontenc}

\usepackage{mathptmx}
\usepackage[scaled=.90]{helvet}
\usepackage{courier}
\usepackage{sectsty}
\usepackage{amssymb}
\usepackage[titles]{tocloft}
\usepackage{doxygen}
\lstset{language=C++,inputencoding=utf8,basicstyle=\footnotesize,breaklines=true,breakatwhitespace=true,tabsize=4,numbers=left }
\makeindex
\setcounter{tocdepth}{3}
\renewcommand{\footrulewidth}{0.4pt}
\renewcommand{\familydefault}{\sfdefault}
\hfuzz=15pt
\setlength{\emergencystretch}{15pt}
\hbadness=750
\tolerance=750
\begin{document}
\hypersetup{pageanchor=false,citecolor=blue}
\begin{titlepage}
\vspace*{7cm}
\begin{center}
{\Large Sortowanie \\[1ex]\large 4.\-0 }\\
\vspace*{1cm}
{\large Wygenerowano przez Doxygen 1.8.3.1}\\
\vspace*{0.5cm}
{\small N, 23 mar 2014 20:17:27}\\
\end{center}
\end{titlepage}
\clearemptydoublepage
\pagenumbering{roman}
\tableofcontents
\clearemptydoublepage
\pagenumbering{arabic}
\hypersetup{pageanchor=true,citecolor=blue}
\chapter{Dokumentacja zadania P\-A\-M\-S\-I L\-A\-B}
\label{index}\hypertarget{index}{}\begin{DoxyDate}{Data}
\-: 30 kwi 2014 
\end{DoxyDate}
\begin{DoxyAuthor}{Autor}
\-: Karolina Morawska 
\end{DoxyAuthor}
\begin{DoxyVersion}{Wersja}
\-: 0.\-1 
\end{DoxyVersion}

\chapter{Indeks klas}
\section{Lista klas}
Tutaj znajdują się klasy, struktury, unie i interfejsy wraz z ich krótkimi opisami\-:\begin{DoxyCompactList}
\item\contentsline{section}{\hyperlink{class_graf}{Graf} \\*Modeluje pojęcie \hyperlink{class_graf}{Graf}. Klasa modeluje pojęcie graf . Jej atrybutem jest pole zawierajace liste sasedztwa }{\pageref{class_graf}}{}
\item\contentsline{section}{\hyperlink{classpara}{para} \\*Modeluje pojęcie para. Klasa modeluje pojęcie para . Jej atrybutem sa pola zawierajace numer wierzcholka i jego wage }{\pageref{classpara}}{}
\end{DoxyCompactList}

\chapter{Indeks plików}
\section{Lista plików}
Tutaj znajduje się lista wszystkich plików z ich krótkimi opisami\-:\begin{DoxyCompactList}
\item\contentsline{section}{\hyperlink{dzialania_8cpp}{dzialania.\-cpp} \\*Metod klasy \hyperlink{class_dzialania}{Dzialania} }{\pageref{dzialania_8cpp}}{}
\item\contentsline{section}{\hyperlink{dzialania_8hh}{dzialania.\-hh} \\*Definicja klasy \hyperlink{class_dzialania}{Dzialania} }{\pageref{dzialania_8hh}}{}
\item\contentsline{section}{\hyperlink{main_8cpp}{main.\-cpp} \\*Glowny plik programu }{\pageref{main_8cpp}}{}
\item\contentsline{section}{\hyperlink{tablica_8cpp}{tablica.\-cpp} \\*Metody klasy \hyperlink{class_tablica}{Tablica} }{\pageref{tablica_8cpp}}{}
\item\contentsline{section}{\hyperlink{tablica_8hh}{tablica.\-hh} \\*Definicja klasy \hyperlink{class_tablica}{Tablica} }{\pageref{tablica_8hh}}{}
\end{DoxyCompactList}

\chapter{Dokumentacja klas}
\hypertarget{class_czaskol}{\section{Dokumentacja szablonu klasy Czaskol$<$ klasakol $>$}
\label{class_czaskol}\index{Czaskol$<$ klasakol $>$@{Czaskol$<$ klasakol $>$}}
}


{\ttfamily \#include $<$czas.\-hh$>$}

\subsection*{Metody publiczne}
\begin{DoxyCompactItemize}
\item 
double \hyperlink{class_czaskol_a0ce6e7012a511af4c9c0012bfdec2fbc}{czas} (unsigned int lpowtorzen, unsigned int zmienna)
\end{DoxyCompactItemize}


\subsection{Opis szczegółowy}
\subsubsection*{template$<$class klasakol$>$class Czaskol$<$ klasakol $>$}



Definicja w linii 21 pliku czas.\-hh.



\subsection{Dokumentacja funkcji składowych}
\hypertarget{class_czaskol_a0ce6e7012a511af4c9c0012bfdec2fbc}{\index{Czaskol@{Czaskol}!czas@{czas}}
\index{czas@{czas}!Czaskol@{Czaskol}}
\subsubsection[{czas}]{\setlength{\rightskip}{0pt plus 5cm}template$<$class klasakol $>$ double {\bf Czaskol}$<$ klasakol $>$\-::czas (
\begin{DoxyParamCaption}
\item[{unsigned int}]{lpowtorzen, }
\item[{unsigned int}]{zmienna}
\end{DoxyParamCaption}
)}}\label{class_czaskol_a0ce6e7012a511af4c9c0012bfdec2fbc}
Funckja liczaca czas dzialania algorytmu kolejki 

Definicja w linii 44 pliku czas.\-hh.



Dokumentacja dla tej klasy została wygenerowana z pliku\-:\begin{DoxyCompactItemize}
\item 
/home/karolina/\-Pulpit/pamsi/prj/inc/\hyperlink{czas_8hh}{czas.\-hh}\end{DoxyCompactItemize}

\hypertarget{class_czasstos}{\section{Dokumentacja szablonu klasy Czasstos$<$ klasasto $>$}
\label{class_czasstos}\index{Czasstos$<$ klasasto $>$@{Czasstos$<$ klasasto $>$}}
}


Sablon klasy \hyperlink{class_czasstos}{Czasstos}. Klasa \hyperlink{class_czasstos}{Czasstos} mierzy czas dzialania algorytmu wykorzystuje $\ast$parametry takie jak \-: liczba powtorzen i zmienna-\/czyli rozmiar problemu .  




{\ttfamily \#include $<$czas.\-hh$>$}

\subsection*{Metody publiczne}
\begin{DoxyCompactItemize}
\item 
double \hyperlink{class_czasstos_a49b08edabe35c166e4f8cddb4c9260c3}{czas} (unsigned int lpowtorzen, unsigned int zmienna)
\end{DoxyCompactItemize}


\subsection{Opis szczegółowy}
\subsubsection*{template$<$class klasasto$>$class Czasstos$<$ klasasto $>$}

Sablon klasy \hyperlink{class_czasstos}{Czasstos}. Klasa \hyperlink{class_czasstos}{Czasstos} mierzy czas dzialania algorytmu wykorzystuje $\ast$parametry takie jak \-: liczba powtorzen i zmienna-\/czyli rozmiar problemu . 

Definicja w linii 16 pliku czas.\-hh.



\subsection{Dokumentacja funkcji składowych}
\hypertarget{class_czasstos_a49b08edabe35c166e4f8cddb4c9260c3}{\index{Czasstos@{Czasstos}!czas@{czas}}
\index{czas@{czas}!Czasstos@{Czasstos}}
\subsubsection[{czas}]{\setlength{\rightskip}{0pt plus 5cm}template$<$class klasasto $>$ double {\bf Czasstos}$<$ klasasto $>$\-::czas (
\begin{DoxyParamCaption}
\item[{unsigned int}]{lpowtorzen, }
\item[{unsigned int}]{zmienna}
\end{DoxyParamCaption}
)}}\label{class_czasstos_a49b08edabe35c166e4f8cddb4c9260c3}
Funckja liczaca czas dzialania algorytmu stosu 

Definicja w linii 30 pliku czas.\-hh.



Oto graf wywoływań tej funkcji\-:\nopagebreak
\begin{figure}[H]
\begin{center}
\leavevmode
\includegraphics[width=240pt]{class_czasstos_a49b08edabe35c166e4f8cddb4c9260c3_icgraph}
\end{center}
\end{figure}




Dokumentacja dla tej klasy została wygenerowana z pliku\-:\begin{DoxyCompactItemize}
\item 
/home/karolina/\-Pulpit/pamsi/prj/inc/\hyperlink{czas_8hh}{czas.\-hh}\end{DoxyCompactItemize}

\hypertarget{class_dzialania}{\section{Dokumentacja klasy Dzialania}
\label{class_dzialania}\index{Dzialania@{Dzialania}}
}


Deklaracja klasy \hyperlink{class_dzialania}{Dzialania}.  




{\ttfamily \#include $<$dzialania.\-hh$>$}

\subsection*{Metody publiczne}
\begin{DoxyCompactItemize}
\item 
bool \hyperlink{class_dzialania_a4d1a1b41a0f2f76d4c16ad20f77b7cfa}{wczytajplik} (char $\ast$nazwapl)
\item 
bool \hyperlink{class_dzialania_af059b80e034854eb1ae878d6b636fff6}{porownaj} (char $\ast$nazwapl)
\item 
double \hyperlink{class_dzialania_a8c13fb89281d74f9dd8dd22f43bffbb9}{liczczas} (int iloscpowtorzen)
\item 
int \hyperlink{class_dzialania_a177e69d16b8280aae1d658adb67a8fbc}{rozmiartab} ()
\end{DoxyCompactItemize}


\subsection{Opis szczegółowy}
Deklaracja klasy \hyperlink{class_dzialania}{Dzialania}. 

Klasa \hyperlink{class_dzialania}{Dzialania} posiada pola oraz funkcje potrzebne do wykonywania dzialan na tablicach . 

Definicja w linii 13 pliku dzialania.\-hh.



\subsection{Dokumentacja funkcji składowych}
\hypertarget{class_dzialania_a8c13fb89281d74f9dd8dd22f43bffbb9}{\index{Dzialania@{Dzialania}!liczczas@{liczczas}}
\index{liczczas@{liczczas}!Dzialania@{Dzialania}}
\subsubsection[{liczczas}]{\setlength{\rightskip}{0pt plus 5cm}double Dzialania\-::liczczas (
\begin{DoxyParamCaption}
\item[{int}]{iloscpowtorzen}
\end{DoxyParamCaption}
)}}\label{class_dzialania_a8c13fb89281d74f9dd8dd22f43bffbb9}
Funkcja mierzy czas dzialania algorytmu.

Argumenty i najwazniejsze pola funkcji -\/iloscpowtorzen -\/zmienna zawierajaca ile powtorzen ma wykonywac program. 

Definicja w linii 80 pliku dzialania.\-cpp.

\hypertarget{class_dzialania_af059b80e034854eb1ae878d6b636fff6}{\index{Dzialania@{Dzialania}!porownaj@{porownaj}}
\index{porownaj@{porownaj}!Dzialania@{Dzialania}}
\subsubsection[{porownaj}]{\setlength{\rightskip}{0pt plus 5cm}bool Dzialania\-::porownaj (
\begin{DoxyParamCaption}
\item[{char $\ast$}]{nazwapl}
\end{DoxyParamCaption}
)}}\label{class_dzialania_af059b80e034854eb1ae878d6b636fff6}
Funkcja porownuje dwa pliki \-: plik wejsciowy i sprawdzajacy , informuje o porawnosci wykonywanego $\ast$ dzialania.

Argumenty i najwazniejsze pola funkcji -\/nazwapl zmienna typu char zawierajaca nazwe pliku 

Definicja w linii 49 pliku dzialania.\-cpp.

\hypertarget{class_dzialania_a177e69d16b8280aae1d658adb67a8fbc}{\index{Dzialania@{Dzialania}!rozmiartab@{rozmiartab}}
\index{rozmiartab@{rozmiartab}!Dzialania@{Dzialania}}
\subsubsection[{rozmiartab}]{\setlength{\rightskip}{0pt plus 5cm}int Dzialania\-::rozmiartab (
\begin{DoxyParamCaption}
{}
\end{DoxyParamCaption}
)\hspace{0.3cm}{\ttfamily [inline]}}}\label{class_dzialania_a177e69d16b8280aae1d658adb67a8fbc}
Funkcja pomocnicza zwraca rozmiar tablicy . 

Definicja w linii 36 pliku dzialania.\-hh.

\hypertarget{class_dzialania_a4d1a1b41a0f2f76d4c16ad20f77b7cfa}{\index{Dzialania@{Dzialania}!wczytajplik@{wczytajplik}}
\index{wczytajplik@{wczytajplik}!Dzialania@{Dzialania}}
\subsubsection[{wczytajplik}]{\setlength{\rightskip}{0pt plus 5cm}bool Dzialania\-::wczytajplik (
\begin{DoxyParamCaption}
\item[{char $\ast$}]{nazwapl}
\end{DoxyParamCaption}
)}}\label{class_dzialania_a4d1a1b41a0f2f76d4c16ad20f77b7cfa}
Funkcja wczytujaca plik i sprawdzajaca porawnosc wykonania funkcji.

Argumenty i najwazniejsze pola funkcji -\/nazwapl zmienna typu char zawierajaca nazwe pliku 

Definicja w linii 24 pliku dzialania.\-cpp.



Dokumentacja dla tej klasy została wygenerowana z plików\-:\begin{DoxyCompactItemize}
\item 
/home/karolina/\-Pulpit/pamsi/prj/inc/\hyperlink{dzialania_8hh}{dzialania.\-hh}\item 
/home/karolina/\-Pulpit/pamsi/prj/src/\hyperlink{dzialania_8cpp}{dzialania.\-cpp}\end{DoxyCompactItemize}

\hypertarget{class_kolejka}{\section{Dokumentacja szablonu klasy Kolejka$<$ Typ $>$}
\label{class_kolejka}\index{Kolejka$<$ Typ $>$@{Kolejka$<$ Typ $>$}}
}


Szablon klasy \hyperlink{class_kolejka}{Kolejka} Klasa zaimplementowana na liscie.  




{\ttfamily \#include $<$kolejka.\-hh$>$}

\subsection*{Metody publiczne}
\begin{DoxyCompactItemize}
\item 
unsigned int \hyperlink{class_kolejka_a6fa67cc293681c3333e6d553cfe3ce84}{size} () const 
\item 
bool \hyperlink{class_kolejka_a6dca0c5c22b17197882b83a54b4649d4}{isempty} () const 
\item 
const Typ \& \hyperlink{class_kolejka_ab0222c8041187d540a4fe63ae15f0799}{top} () const 
\item 
void \hyperlink{class_kolejka_ac30514f0d3eb95411b472baebe859f2e}{enqueue} (const Typ element)
\item 
void \hyperlink{class_kolejka_a73e7c4df8f400108a4c48132c2476d5a}{dequeue} ()
\end{DoxyCompactItemize}


\subsection{Opis szczegółowy}
\subsubsection*{template$<$typename Typ$>$class Kolejka$<$ Typ $>$}

Szablon klasy \hyperlink{class_kolejka}{Kolejka} Klasa zaimplementowana na liscie. 



Definicja w linii 10 pliku kolejka.\-hh.



\subsection{Dokumentacja funkcji składowych}
\hypertarget{class_kolejka_a73e7c4df8f400108a4c48132c2476d5a}{\index{Kolejka@{Kolejka}!dequeue@{dequeue}}
\index{dequeue@{dequeue}!Kolejka@{Kolejka}}
\subsubsection[{dequeue}]{\setlength{\rightskip}{0pt plus 5cm}template$<$typename Typ $>$ void {\bf Kolejka}$<$ Typ $>$\-::dequeue (
\begin{DoxyParamCaption}
{}
\end{DoxyParamCaption}
)\hspace{0.3cm}{\ttfamily [inline]}}}\label{class_kolejka_a73e7c4df8f400108a4c48132c2476d5a}
Metoda ktora sciaga element ze stosu 

Definicja w linii 33 pliku kolejka.\-hh.

\hypertarget{class_kolejka_ac30514f0d3eb95411b472baebe859f2e}{\index{Kolejka@{Kolejka}!enqueue@{enqueue}}
\index{enqueue@{enqueue}!Kolejka@{Kolejka}}
\subsubsection[{enqueue}]{\setlength{\rightskip}{0pt plus 5cm}template$<$typename Typ $>$ void {\bf Kolejka}$<$ Typ $>$\-::enqueue (
\begin{DoxyParamCaption}
\item[{const Typ}]{element}
\end{DoxyParamCaption}
)\hspace{0.3cm}{\ttfamily [inline]}}}\label{class_kolejka_ac30514f0d3eb95411b472baebe859f2e}
Metoda ktora dodaje do kolejki elementy 

Definicja w linii 28 pliku kolejka.\-hh.

\hypertarget{class_kolejka_a6dca0c5c22b17197882b83a54b4649d4}{\index{Kolejka@{Kolejka}!isempty@{isempty}}
\index{isempty@{isempty}!Kolejka@{Kolejka}}
\subsubsection[{isempty}]{\setlength{\rightskip}{0pt plus 5cm}template$<$typename Typ $>$ bool {\bf Kolejka}$<$ Typ $>$\-::isempty (
\begin{DoxyParamCaption}
{}
\end{DoxyParamCaption}
) const\hspace{0.3cm}{\ttfamily [inline]}}}\label{class_kolejka_a6dca0c5c22b17197882b83a54b4649d4}
Metoda sprawdza zawartosc listy 

Definicja w linii 20 pliku kolejka.\-hh.

\hypertarget{class_kolejka_a6fa67cc293681c3333e6d553cfe3ce84}{\index{Kolejka@{Kolejka}!size@{size}}
\index{size@{size}!Kolejka@{Kolejka}}
\subsubsection[{size}]{\setlength{\rightskip}{0pt plus 5cm}template$<$typename Typ $>$ unsigned int {\bf Kolejka}$<$ Typ $>$\-::size (
\begin{DoxyParamCaption}
{}
\end{DoxyParamCaption}
) const\hspace{0.3cm}{\ttfamily [inline]}}}\label{class_kolejka_a6fa67cc293681c3333e6d553cfe3ce84}
Metoda zwraca rozmiar listy 

Definicja w linii 17 pliku kolejka.\-hh.

\hypertarget{class_kolejka_ab0222c8041187d540a4fe63ae15f0799}{\index{Kolejka@{Kolejka}!top@{top}}
\index{top@{top}!Kolejka@{Kolejka}}
\subsubsection[{top}]{\setlength{\rightskip}{0pt plus 5cm}template$<$typename Typ $>$ const Typ\& {\bf Kolejka}$<$ Typ $>$\-::top (
\begin{DoxyParamCaption}
{}
\end{DoxyParamCaption}
) const\hspace{0.3cm}{\ttfamily [inline]}}}\label{class_kolejka_ab0222c8041187d540a4fe63ae15f0799}
Metoda sprawdzajaca wierzcholek 

Definicja w linii 24 pliku kolejka.\-hh.



Dokumentacja dla tej klasy została wygenerowana z pliku\-:\begin{DoxyCompactItemize}
\item 
/home/karolina/\-Pulpit/pamsi/prj/inc/\hyperlink{kolejka_8hh}{kolejka.\-hh}\end{DoxyCompactItemize}

\hypertarget{class_stol}{\section{Dokumentacja szablonu klasy Stol$<$ Typ $>$}
\label{class_stol}\index{Stol$<$ Typ $>$@{Stol$<$ Typ $>$}}
}


Szablon klasy \hyperlink{class_stol}{Stol}.  




{\ttfamily \#include $<$stoslista.\-hh$>$}

\subsection*{Metody publiczne}
\begin{DoxyCompactItemize}
\item 
unsigned int \hyperlink{class_stol_a0fe3754424071aaa5a8bcd45fd145c5b}{size} () const 
\item 
void \hyperlink{class_stol_a6a92606f32fb4bed2261b7ca5e192e4c}{push} (Typ element)
\item 
void \hyperlink{class_stol_ab6de7dccb1eb7324c8a9b866fc557ed6}{pop} ()
\item 
bool \hyperlink{class_stol_ab4ffd8ef21aefe4966b65ba923b0b780}{isempty} () const 
\item 
const Typ \& \hyperlink{class_stol_a347f84a10b44bc97af0d6f77643cd45a}{top} () const 
\end{DoxyCompactItemize}
\subsection*{Atrybuty prywatne}
\begin{DoxyCompactItemize}
\item 
std\-::list$<$ Typ $>$ \hyperlink{class_stol_a1684a3d6a3801a6e4ea2ade79e634ea2}{lista}
\end{DoxyCompactItemize}


\subsection{Opis szczegółowy}
\subsubsection*{template$<$typename Typ$>$class Stol$<$ Typ $>$}

Klasa \hyperlink{class_stol}{Stol} posiada pola oraz funkcje potrzebne do wykonywania dzialan na liscie. 

Definicja w linii 11 pliku stoslista.\-hh.



\subsection{Dokumentacja funkcji składowych}
\hypertarget{class_stol_ab4ffd8ef21aefe4966b65ba923b0b780}{\index{Stol@{Stol}!isempty@{isempty}}
\index{isempty@{isempty}!Stol@{Stol}}
\subsubsection[{isempty}]{\setlength{\rightskip}{0pt plus 5cm}template$<$typename Typ $>$ bool {\bf Stol}$<$ Typ $>$\-::isempty (
\begin{DoxyParamCaption}
{}
\end{DoxyParamCaption}
) const\hspace{0.3cm}{\ttfamily [inline]}}}\label{class_stol_ab4ffd8ef21aefe4966b65ba923b0b780}
Metoda sprawdza zawartosc listy 

Definicja w linii 30 pliku stoslista.\-hh.

\hypertarget{class_stol_ab6de7dccb1eb7324c8a9b866fc557ed6}{\index{Stol@{Stol}!pop@{pop}}
\index{pop@{pop}!Stol@{Stol}}
\subsubsection[{pop}]{\setlength{\rightskip}{0pt plus 5cm}template$<$typename Typ $>$ void {\bf Stol}$<$ Typ $>$\-::pop (
\begin{DoxyParamCaption}
{}
\end{DoxyParamCaption}
)\hspace{0.3cm}{\ttfamily [inline]}}}\label{class_stol_ab6de7dccb1eb7324c8a9b866fc557ed6}
Metoda sciaga element z listy 

Definicja w linii 25 pliku stoslista.\-hh.

\hypertarget{class_stol_a6a92606f32fb4bed2261b7ca5e192e4c}{\index{Stol@{Stol}!push@{push}}
\index{push@{push}!Stol@{Stol}}
\subsubsection[{push}]{\setlength{\rightskip}{0pt plus 5cm}template$<$typename Typ $>$ void {\bf Stol}$<$ Typ $>$\-::push (
\begin{DoxyParamCaption}
\item[{Typ}]{element}
\end{DoxyParamCaption}
)\hspace{0.3cm}{\ttfamily [inline]}}}\label{class_stol_a6a92606f32fb4bed2261b7ca5e192e4c}
Metoda dodaje element 

Definicja w linii 20 pliku stoslista.\-hh.

\hypertarget{class_stol_a0fe3754424071aaa5a8bcd45fd145c5b}{\index{Stol@{Stol}!size@{size}}
\index{size@{size}!Stol@{Stol}}
\subsubsection[{size}]{\setlength{\rightskip}{0pt plus 5cm}template$<$typename Typ $>$ unsigned int {\bf Stol}$<$ Typ $>$\-::size (
\begin{DoxyParamCaption}
{}
\end{DoxyParamCaption}
) const\hspace{0.3cm}{\ttfamily [inline]}}}\label{class_stol_a0fe3754424071aaa5a8bcd45fd145c5b}
Metoda zwraca rozmiar listy 

Definicja w linii 18 pliku stoslista.\-hh.

\hypertarget{class_stol_a347f84a10b44bc97af0d6f77643cd45a}{\index{Stol@{Stol}!top@{top}}
\index{top@{top}!Stol@{Stol}}
\subsubsection[{top}]{\setlength{\rightskip}{0pt plus 5cm}template$<$typename Typ $>$ const Typ\& {\bf Stol}$<$ Typ $>$\-::top (
\begin{DoxyParamCaption}
{}
\end{DoxyParamCaption}
) const\hspace{0.3cm}{\ttfamily [inline]}}}\label{class_stol_a347f84a10b44bc97af0d6f77643cd45a}
Metoda sprawdza wierzchlek stosu. 

Definicja w linii 34 pliku stoslista.\-hh.



\subsection{Dokumentacja atrybutów składowych}
\hypertarget{class_stol_a1684a3d6a3801a6e4ea2ade79e634ea2}{\index{Stol@{Stol}!lista@{lista}}
\index{lista@{lista}!Stol@{Stol}}
\subsubsection[{lista}]{\setlength{\rightskip}{0pt plus 5cm}template$<$typename Typ $>$ std\-::list$<$Typ$>$ {\bf Stol}$<$ Typ $>$\-::lista\hspace{0.3cm}{\ttfamily [private]}}}\label{class_stol_a1684a3d6a3801a6e4ea2ade79e634ea2}


Definicja w linii 13 pliku stoslista.\-hh.



Dokumentacja dla tej klasy została wygenerowana z pliku\-:\begin{DoxyCompactItemize}
\item 
\hyperlink{stoslista_8hh}{stoslista.\-hh}\end{DoxyCompactItemize}

\hypertarget{class_stos}{\section{Dokumentacja szablonu klasy Stos$<$ Typ $>$}
\label{class_stos}\index{Stos$<$ Typ $>$@{Stos$<$ Typ $>$}}
}


Szablon klasy stos. Klasa stos wykonuja dzialania \-: -\/odklada elementy na stos -\/sciaga element ze stosu -\/zwraca rozmiar -\/sprawdza czy stos jesyt pusty.  




{\ttfamily \#include $<$stos.\-hh$>$}

\subsection*{Metody publiczne}
\begin{DoxyCompactItemize}
\item 
const Typ \& \hyperlink{class_stos_a07d6df2e59cdfdac65d8f793777ff463}{top} () const 
\item 
void \hyperlink{class_stos_ae288acf023c8a1c25184dc5c6a57ee2a}{push} (Typ element)
\item 
void \hyperlink{class_stos_a38e84abe604aa6d1e9b9c6ac524a33bc}{pop} ()
\item 
bool \hyperlink{class_stos_a70aadb72c5b4728dc0c4b1e1adaa868e}{isempty} () const 
\item 
int \hyperlink{class_stos_a472d836bf51faa0264fefd08eb4224d6}{size} () const 
\end{DoxyCompactItemize}
\subsection*{Atrybuty prywatne}
\begin{DoxyCompactItemize}
\item 
\hyperlink{class_tablica}{Tablica}$<$ Typ $>$ \hyperlink{class_stos_abda1a3c37f21b8728a3fcfb4accd2108}{tab}
\begin{DoxyCompactList}\small\item\em Pole przechowujace tablic. \end{DoxyCompactList}\end{DoxyCompactItemize}


\subsection{Opis szczegółowy}
\subsubsection*{template$<$typename Typ$>$class Stos$<$ Typ $>$}



Definicja w linii 15 pliku stos.\-hh.



\subsection{Dokumentacja funkcji składowych}
\hypertarget{class_stos_a70aadb72c5b4728dc0c4b1e1adaa868e}{\index{Stos@{Stos}!isempty@{isempty}}
\index{isempty@{isempty}!Stos@{Stos}}
\subsubsection[{isempty}]{\setlength{\rightskip}{0pt plus 5cm}template$<$typename Typ $>$ bool {\bf Stos}$<$ Typ $>$\-::isempty (
\begin{DoxyParamCaption}
{}
\end{DoxyParamCaption}
) const\hspace{0.3cm}{\ttfamily [inline]}}}\label{class_stos_a70aadb72c5b4728dc0c4b1e1adaa868e}
Metoda sprawdza czy stos jest pusty. 

Definicja w linii 29 pliku stos.\-hh.

\hypertarget{class_stos_a38e84abe604aa6d1e9b9c6ac524a33bc}{\index{Stos@{Stos}!pop@{pop}}
\index{pop@{pop}!Stos@{Stos}}
\subsubsection[{pop}]{\setlength{\rightskip}{0pt plus 5cm}template$<$typename Typ $>$ void {\bf Stos}$<$ Typ $>$\-::pop (
\begin{DoxyParamCaption}
{}
\end{DoxyParamCaption}
)}}\label{class_stos_a38e84abe604aa6d1e9b9c6ac524a33bc}
Metoda sciaga ze stosu element. 

Definicja w linii 46 pliku stos.\-hh.

\hypertarget{class_stos_ae288acf023c8a1c25184dc5c6a57ee2a}{\index{Stos@{Stos}!push@{push}}
\index{push@{push}!Stos@{Stos}}
\subsubsection[{push}]{\setlength{\rightskip}{0pt plus 5cm}template$<$typename Typ $>$ void {\bf Stos}$<$ Typ $>$\-::push (
\begin{DoxyParamCaption}
\item[{Typ}]{element}
\end{DoxyParamCaption}
)}}\label{class_stos_ae288acf023c8a1c25184dc5c6a57ee2a}
Metoda odkladajaca na stosie elementy 

Definicja w linii 41 pliku stos.\-hh.

\hypertarget{class_stos_a472d836bf51faa0264fefd08eb4224d6}{\index{Stos@{Stos}!size@{size}}
\index{size@{size}!Stos@{Stos}}
\subsubsection[{size}]{\setlength{\rightskip}{0pt plus 5cm}template$<$typename Typ $>$ int {\bf Stos}$<$ Typ $>$\-::size (
\begin{DoxyParamCaption}
{}
\end{DoxyParamCaption}
) const\hspace{0.3cm}{\ttfamily [inline]}}}\label{class_stos_a472d836bf51faa0264fefd08eb4224d6}
Metoda zwraca rozmiar stosu. 

Definicja w linii 34 pliku stos.\-hh.

\hypertarget{class_stos_a07d6df2e59cdfdac65d8f793777ff463}{\index{Stos@{Stos}!top@{top}}
\index{top@{top}!Stos@{Stos}}
\subsubsection[{top}]{\setlength{\rightskip}{0pt plus 5cm}template$<$typename Typ $>$ const Typ\& {\bf Stos}$<$ Typ $>$\-::top (
\begin{DoxyParamCaption}
{}
\end{DoxyParamCaption}
) const\hspace{0.3cm}{\ttfamily [inline]}}}\label{class_stos_a07d6df2e59cdfdac65d8f793777ff463}
Metoda zwraca szczyt stosu 

Definicja w linii 20 pliku stos.\-hh.



\subsection{Dokumentacja atrybutów składowych}
\hypertarget{class_stos_abda1a3c37f21b8728a3fcfb4accd2108}{\index{Stos@{Stos}!tab@{tab}}
\index{tab@{tab}!Stos@{Stos}}
\subsubsection[{tab}]{\setlength{\rightskip}{0pt plus 5cm}template$<$typename Typ $>$ {\bf Tablica}$<$Typ$>$ {\bf Stos}$<$ Typ $>$\-::tab\hspace{0.3cm}{\ttfamily [private]}}}\label{class_stos_abda1a3c37f21b8728a3fcfb4accd2108}


Definicja w linii 37 pliku stos.\-hh.



Dokumentacja dla tej klasy została wygenerowana z pliku\-:\begin{DoxyCompactItemize}
\item 
\hyperlink{stos_8hh}{stos.\-hh}\end{DoxyCompactItemize}

\hypertarget{class_stos2}{\section{Dokumentacja szablonu klasy Stos2$<$ Typ $>$}
\label{class_stos2}\index{Stos2$<$ Typ $>$@{Stos2$<$ Typ $>$}}
}


Szablon klasy \hyperlink{class_stos2}{Stos2} \hyperlink{class_stos}{Stos} ktory w przypadku przepelnienia zmienia swoj rozmiar. Metody analogiczne do klasy \hyperlink{class_stos}{Stos}.  




{\ttfamily \#include $<$stos2.\-hh$>$}

\subsection*{Metody publiczne}
\begin{DoxyCompactItemize}
\item 
const Typ \& \hyperlink{class_stos2_a4cefb452cfd2da4eab15b9380c87b219}{top} () const 
\item 
void \hyperlink{class_stos2_ab9bf4dc27877d77a3a53736d33fb965f}{push} (Typ element)
\item 
void \hyperlink{class_stos2_a9695a855dac219cc685cdded92ee38ac}{pop} ()
\item 
bool \hyperlink{class_stos2_a749e533abbbb8db8e5d97e6028869a7a}{isempty} () const 
\item 
unsigned int \hyperlink{class_stos2_a6599f822eabbade13ab6593ebec18182}{size} () const 
\item 
\hyperlink{class_stos2_a91771fcaa090ac82d4e14603624f4dd3}{Stos2} ()
\end{DoxyCompactItemize}
\subsection*{Atrybuty prywatne}
\begin{DoxyCompactItemize}
\item 
\hyperlink{class_tablica}{Tablica}$<$ Typ $>$ \hyperlink{class_stos2_a9c7f4132f7a4302226d62fd816ed10ed}{tab}
\item 
unsigned int \hyperlink{class_stos2_ad485021174de06d5faedba7cad9cbaa2}{dltab}
\end{DoxyCompactItemize}


\subsection{Opis szczegółowy}
\subsubsection*{template$<$typename Typ$>$class Stos2$<$ Typ $>$}



Definicja w linii 11 pliku stos2.\-hh.



\subsection{Dokumentacja konstruktora i destruktora}
\hypertarget{class_stos2_a91771fcaa090ac82d4e14603624f4dd3}{\index{Stos2@{Stos2}!Stos2@{Stos2}}
\index{Stos2@{Stos2}!Stos2@{Stos2}}
\subsubsection[{Stos2}]{\setlength{\rightskip}{0pt plus 5cm}template$<$typename Typ $>$ {\bf Stos2}$<$ Typ $>$\-::{\bf Stos2} (
\begin{DoxyParamCaption}
{}
\end{DoxyParamCaption}
)\hspace{0.3cm}{\ttfamily [inline]}}}\label{class_stos2_a91771fcaa090ac82d4e14603624f4dd3}
Konstruktor 

Definicja w linii 33 pliku stos2.\-hh.



\subsection{Dokumentacja funkcji składowych}
\hypertarget{class_stos2_a749e533abbbb8db8e5d97e6028869a7a}{\index{Stos2@{Stos2}!isempty@{isempty}}
\index{isempty@{isempty}!Stos2@{Stos2}}
\subsubsection[{isempty}]{\setlength{\rightskip}{0pt plus 5cm}template$<$typename Typ $>$ bool {\bf Stos2}$<$ Typ $>$\-::isempty (
\begin{DoxyParamCaption}
{}
\end{DoxyParamCaption}
) const\hspace{0.3cm}{\ttfamily [inline]}}}\label{class_stos2_a749e533abbbb8db8e5d97e6028869a7a}
Metoda sprawdza czy stos jest pusty. 

Definicja w linii 25 pliku stos2.\-hh.

\hypertarget{class_stos2_a9695a855dac219cc685cdded92ee38ac}{\index{Stos2@{Stos2}!pop@{pop}}
\index{pop@{pop}!Stos2@{Stos2}}
\subsubsection[{pop}]{\setlength{\rightskip}{0pt plus 5cm}template$<$typename Typ $>$ void {\bf Stos2}$<$ Typ $>$\-::pop (
\begin{DoxyParamCaption}
{}
\end{DoxyParamCaption}
)}}\label{class_stos2_a9695a855dac219cc685cdded92ee38ac}
Metoda sciaga ze stosu element. 

Definicja w linii 52 pliku stos2.\-hh.

\hypertarget{class_stos2_ab9bf4dc27877d77a3a53736d33fb965f}{\index{Stos2@{Stos2}!push@{push}}
\index{push@{push}!Stos2@{Stos2}}
\subsubsection[{push}]{\setlength{\rightskip}{0pt plus 5cm}template$<$typename Typ $>$ void {\bf Stos2}$<$ Typ $>$\-::push (
\begin{DoxyParamCaption}
\item[{Typ}]{element}
\end{DoxyParamCaption}
)}}\label{class_stos2_ab9bf4dc27877d77a3a53736d33fb965f}
Metoda odkladajaca na stosie elementy 

Definicja w linii 42 pliku stos2.\-hh.

\hypertarget{class_stos2_a6599f822eabbade13ab6593ebec18182}{\index{Stos2@{Stos2}!size@{size}}
\index{size@{size}!Stos2@{Stos2}}
\subsubsection[{size}]{\setlength{\rightskip}{0pt plus 5cm}template$<$typename Typ $>$ unsigned int {\bf Stos2}$<$ Typ $>$\-::size (
\begin{DoxyParamCaption}
{}
\end{DoxyParamCaption}
) const\hspace{0.3cm}{\ttfamily [inline]}}}\label{class_stos2_a6599f822eabbade13ab6593ebec18182}
Metoda zwraca rozmiar stosu. 

Definicja w linii 30 pliku stos2.\-hh.

\hypertarget{class_stos2_a4cefb452cfd2da4eab15b9380c87b219}{\index{Stos2@{Stos2}!top@{top}}
\index{top@{top}!Stos2@{Stos2}}
\subsubsection[{top}]{\setlength{\rightskip}{0pt plus 5cm}template$<$typename Typ $>$ const Typ\& {\bf Stos2}$<$ Typ $>$\-::top (
\begin{DoxyParamCaption}
{}
\end{DoxyParamCaption}
) const\hspace{0.3cm}{\ttfamily [inline]}}}\label{class_stos2_a4cefb452cfd2da4eab15b9380c87b219}
Metoda odczytujaca wierzcholek stosu 

Definicja w linii 16 pliku stos2.\-hh.



\subsection{Dokumentacja atrybutów składowych}
\hypertarget{class_stos2_ad485021174de06d5faedba7cad9cbaa2}{\index{Stos2@{Stos2}!dltab@{dltab}}
\index{dltab@{dltab}!Stos2@{Stos2}}
\subsubsection[{dltab}]{\setlength{\rightskip}{0pt plus 5cm}template$<$typename Typ $>$ unsigned int {\bf Stos2}$<$ Typ $>$\-::dltab\hspace{0.3cm}{\ttfamily [private]}}}\label{class_stos2_ad485021174de06d5faedba7cad9cbaa2}


Definicja w linii 38 pliku stos2.\-hh.

\hypertarget{class_stos2_a9c7f4132f7a4302226d62fd816ed10ed}{\index{Stos2@{Stos2}!tab@{tab}}
\index{tab@{tab}!Stos2@{Stos2}}
\subsubsection[{tab}]{\setlength{\rightskip}{0pt plus 5cm}template$<$typename Typ $>$ {\bf Tablica}$<$Typ$>$ {\bf Stos2}$<$ Typ $>$\-::tab\hspace{0.3cm}{\ttfamily [private]}}}\label{class_stos2_a9c7f4132f7a4302226d62fd816ed10ed}


Definicja w linii 37 pliku stos2.\-hh.



Dokumentacja dla tej klasy została wygenerowana z pliku\-:\begin{DoxyCompactItemize}
\item 
\hyperlink{stos2_8hh}{stos2.\-hh}\end{DoxyCompactItemize}

\hypertarget{class_tablica}{\section{Dokumentacja klasy Tablica}
\label{class_tablica}\index{Tablica@{Tablica}}
}


Deklaracja klasy \hyperlink{class_tablica}{Tablica}.  




{\ttfamily \#include $<$tablica.\-hh$>$}

\subsection*{Metody publiczne}
\begin{DoxyCompactItemize}
\item 
void \hyperlink{class_tablica_a4f69d95776f0ea1454a87bb72562713b}{zamienelementy} (int i, int j)
\item 
void \hyperlink{class_tablica_ad4d99dc2ca07689167d703ba24a4dab2}{dodajelement} (int element)
\item 
void \hyperlink{class_tablica_ae63b8d381eb4f6a19cae52241228ae07}{odwrockolejnosc} ()
\item 
void \hyperlink{class_tablica_ac5b21c0e98c4f5ac5c728b99f092b112}{dodajelementy} (const \hyperlink{class_tablica}{Tablica} \&T1)
\item 
\hyperlink{class_tablica_a5f484e7b0478e1ff9b62e894f9d7b28d}{Tablica} ()
\item 
\hyperlink{class_tablica_a6e1e50608ad0f9f9626d0b1fb698b180}{$\sim$\-Tablica} ()
\item 
unsigned int \hyperlink{class_tablica_ae95a62ea4245e732b96c110c0fc53532}{rozmiar} () const 
\item 
void \hyperlink{class_tablica_a4e743bdbb74647717d63015894dfae8d}{zmianarozmiaru} (unsigned int nowyrozmiar)
\item 
int \& \hyperlink{class_tablica_aa73e557bfd1a0283d94b594f159cf6d1}{operator\mbox{[}$\,$\mbox{]}} (const unsigned int b)
\item 
const int \& \hyperlink{class_tablica_a16a2591adbcce8add22be48ff8f1a830}{operator\mbox{[}$\,$\mbox{]}} (const unsigned int b) const 
\item 
\hyperlink{class_tablica}{Tablica} \& \hyperlink{class_tablica_a469a403559ecce37e70c8adc67a0fc0d}{operator+} (const \hyperlink{class_tablica}{Tablica} \&argument) const 
\item 
\hyperlink{class_tablica}{Tablica} \& \hyperlink{class_tablica_af42a963c962250d20240811a0defd6b4}{operator=} (const \hyperlink{class_tablica}{Tablica} \&argument)
\item 
bool \hyperlink{class_tablica_a9416cdb689731ec64d440de10d944549}{operator==} (const \hyperlink{class_tablica}{Tablica} \&argument) const 
\end{DoxyCompactItemize}


\subsection{Opis szczegółowy}
Deklaracja klasy \hyperlink{class_tablica}{Tablica}. 

Klasa \hyperlink{class_tablica}{Tablica} posiada pola oraz funkcje potrzebne do wykonywania dzialan na tablicach. 

Definicja w linii 10 pliku tablica.\-hh.



\subsection{Dokumentacja konstruktora i destruktora}
\hypertarget{class_tablica_a5f484e7b0478e1ff9b62e894f9d7b28d}{\index{Tablica@{Tablica}!Tablica@{Tablica}}
\index{Tablica@{Tablica}!Tablica@{Tablica}}
\subsubsection[{Tablica}]{\setlength{\rightskip}{0pt plus 5cm}Tablica\-::\-Tablica (
\begin{DoxyParamCaption}
{}
\end{DoxyParamCaption}
)\hspace{0.3cm}{\ttfamily [inline]}}}\label{class_tablica_a5f484e7b0478e1ff9b62e894f9d7b28d}
Konstruktor klasy \hyperlink{class_tablica}{Tablica}. 

Definicja w linii 32 pliku tablica.\-hh.

\hypertarget{class_tablica_a6e1e50608ad0f9f9626d0b1fb698b180}{\index{Tablica@{Tablica}!$\sim$\-Tablica@{$\sim$\-Tablica}}
\index{$\sim$\-Tablica@{$\sim$\-Tablica}!Tablica@{Tablica}}
\subsubsection[{$\sim$\-Tablica}]{\setlength{\rightskip}{0pt plus 5cm}Tablica\-::$\sim$\-Tablica (
\begin{DoxyParamCaption}
{}
\end{DoxyParamCaption}
)\hspace{0.3cm}{\ttfamily [inline]}}}\label{class_tablica_a6e1e50608ad0f9f9626d0b1fb698b180}
Destruktor klasy \hyperlink{class_tablica}{Tablica}. 

Definicja w linii 35 pliku tablica.\-hh.



\subsection{Dokumentacja funkcji składowych}
\hypertarget{class_tablica_ad4d99dc2ca07689167d703ba24a4dab2}{\index{Tablica@{Tablica}!dodajelement@{dodajelement}}
\index{dodajelement@{dodajelement}!Tablica@{Tablica}}
\subsubsection[{dodajelement}]{\setlength{\rightskip}{0pt plus 5cm}void Tablica\-::dodajelement (
\begin{DoxyParamCaption}
\item[{int}]{element}
\end{DoxyParamCaption}
)}}\label{class_tablica_ad4d99dc2ca07689167d703ba24a4dab2}
Funkcja dodaje element na tablice.\-Wykorzystuje funkcje pomocnicza $\ast$ zmiana rozmiaru.

Argumenty i najwazniejsze pola funkcji -\/element -\/zmienna ktora zostaje dodana do tablicy 

Definicja w linii 44 pliku tablica.\-cpp.

\hypertarget{class_tablica_ac5b21c0e98c4f5ac5c728b99f092b112}{\index{Tablica@{Tablica}!dodajelementy@{dodajelementy}}
\index{dodajelementy@{dodajelementy}!Tablica@{Tablica}}
\subsubsection[{dodajelementy}]{\setlength{\rightskip}{0pt plus 5cm}void Tablica\-::dodajelementy (
\begin{DoxyParamCaption}
\item[{const {\bf Tablica} \&}]{T1}
\end{DoxyParamCaption}
)}}\label{class_tablica_ac5b21c0e98c4f5ac5c728b99f092b112}
Funkcja laczy ze soba dwie tablice.

Argumenty i najwazniejsze pola funkcji -\/\-T1 -\/tablica ktora powtaje po polaczeniu. 

Definicja w linii 53 pliku tablica.\-cpp.

\hypertarget{class_tablica_ae63b8d381eb4f6a19cae52241228ae07}{\index{Tablica@{Tablica}!odwrockolejnosc@{odwrockolejnosc}}
\index{odwrockolejnosc@{odwrockolejnosc}!Tablica@{Tablica}}
\subsubsection[{odwrockolejnosc}]{\setlength{\rightskip}{0pt plus 5cm}void Tablica\-::odwrockolejnosc (
\begin{DoxyParamCaption}
{}
\end{DoxyParamCaption}
)}}\label{class_tablica_ae63b8d381eb4f6a19cae52241228ae07}
Funkcja odwraca kolejnosc w tablicy.

Argumenty i najwazniejsze pola funkcji -\/dlugosctab -\/zmmienna zawierajaca dlugosc tablicy na ktorej wykonywane jest dzialania 

Definicja w linii 22 pliku tablica.\-cpp.

\hypertarget{class_tablica_a469a403559ecce37e70c8adc67a0fc0d}{\index{Tablica@{Tablica}!operator+@{operator+}}
\index{operator+@{operator+}!Tablica@{Tablica}}
\subsubsection[{operator+}]{\setlength{\rightskip}{0pt plus 5cm}{\bf Tablica} \& Tablica\-::operator+ (
\begin{DoxyParamCaption}
\item[{const {\bf Tablica} \&}]{argument}
\end{DoxyParamCaption}
) const}}\label{class_tablica_a469a403559ecce37e70c8adc67a0fc0d}
Przeciazenie operatora dodawania.

Argumenty i najwazniejsze pola funkcji -\/argument -\/przecciazenie operatora 

Definicja w linii 77 pliku tablica.\-cpp.

\hypertarget{class_tablica_af42a963c962250d20240811a0defd6b4}{\index{Tablica@{Tablica}!operator=@{operator=}}
\index{operator=@{operator=}!Tablica@{Tablica}}
\subsubsection[{operator=}]{\setlength{\rightskip}{0pt plus 5cm}{\bf Tablica} \& Tablica\-::operator= (
\begin{DoxyParamCaption}
\item[{const {\bf Tablica} \&}]{argument}
\end{DoxyParamCaption}
)}}\label{class_tablica_af42a963c962250d20240811a0defd6b4}
Przeciazenie operatora przypisywania.

Argumenty i najwazniejsze pola funkcji -\/agrument -\/przeciazenie operatora 

Definicja w linii 65 pliku tablica.\-cpp.

\hypertarget{class_tablica_a9416cdb689731ec64d440de10d944549}{\index{Tablica@{Tablica}!operator==@{operator==}}
\index{operator==@{operator==}!Tablica@{Tablica}}
\subsubsection[{operator==}]{\setlength{\rightskip}{0pt plus 5cm}bool Tablica\-::operator== (
\begin{DoxyParamCaption}
\item[{const {\bf Tablica} \&}]{argument}
\end{DoxyParamCaption}
) const}}\label{class_tablica_a9416cdb689731ec64d440de10d944549}
Przeciazenie operatora porownania.

Argumenty i najwazniejsze pola funkcji -\/argumenti -\/przeciazenie operatora 

Definicja w linii 89 pliku tablica.\-cpp.

\hypertarget{class_tablica_aa73e557bfd1a0283d94b594f159cf6d1}{\index{Tablica@{Tablica}!operator\mbox{[}$\,$\mbox{]}@{operator[]}}
\index{operator\mbox{[}$\,$\mbox{]}@{operator[]}!Tablica@{Tablica}}
\subsubsection[{operator[]}]{\setlength{\rightskip}{0pt plus 5cm}int\& Tablica\-::operator\mbox{[}$\,$\mbox{]} (
\begin{DoxyParamCaption}
\item[{const unsigned int}]{b}
\end{DoxyParamCaption}
)\hspace{0.3cm}{\ttfamily [inline]}}}\label{class_tablica_aa73e557bfd1a0283d94b594f159cf6d1}
Przeciazenie operatora indeksujacego. 

Definicja w linii 44 pliku tablica.\-hh.

\hypertarget{class_tablica_a16a2591adbcce8add22be48ff8f1a830}{\index{Tablica@{Tablica}!operator\mbox{[}$\,$\mbox{]}@{operator[]}}
\index{operator\mbox{[}$\,$\mbox{]}@{operator[]}!Tablica@{Tablica}}
\subsubsection[{operator[]}]{\setlength{\rightskip}{0pt plus 5cm}const int\& Tablica\-::operator\mbox{[}$\,$\mbox{]} (
\begin{DoxyParamCaption}
\item[{const unsigned int}]{b}
\end{DoxyParamCaption}
) const\hspace{0.3cm}{\ttfamily [inline]}}}\label{class_tablica_a16a2591adbcce8add22be48ff8f1a830}


Definicja w linii 45 pliku tablica.\-hh.

\hypertarget{class_tablica_ae95a62ea4245e732b96c110c0fc53532}{\index{Tablica@{Tablica}!rozmiar@{rozmiar}}
\index{rozmiar@{rozmiar}!Tablica@{Tablica}}
\subsubsection[{rozmiar}]{\setlength{\rightskip}{0pt plus 5cm}unsigned int Tablica\-::rozmiar (
\begin{DoxyParamCaption}
{}
\end{DoxyParamCaption}
) const\hspace{0.3cm}{\ttfamily [inline]}}}\label{class_tablica_ae95a62ea4245e732b96c110c0fc53532}
Funkcja pomocnicza zwraca dlugosc tablicy. 

Definicja w linii 38 pliku tablica.\-hh.

\hypertarget{class_tablica_a4f69d95776f0ea1454a87bb72562713b}{\index{Tablica@{Tablica}!zamienelementy@{zamienelementy}}
\index{zamienelementy@{zamienelementy}!Tablica@{Tablica}}
\subsubsection[{zamienelementy}]{\setlength{\rightskip}{0pt plus 5cm}void Tablica\-::zamienelementy (
\begin{DoxyParamCaption}
\item[{int}]{i, }
\item[{int}]{j}
\end{DoxyParamCaption}
)}}\label{class_tablica_a4f69d95776f0ea1454a87bb72562713b}
Funkcja sluzaca do zamiany elementow.

Argumenty i najwazniejsze pola funkcji -\/i, j -\/zmienne oznaczajace elementy w tablicy 

Definicja w linii 9 pliku tablica.\-cpp.

\hypertarget{class_tablica_a4e743bdbb74647717d63015894dfae8d}{\index{Tablica@{Tablica}!zmianarozmiaru@{zmianarozmiaru}}
\index{zmianarozmiaru@{zmianarozmiaru}!Tablica@{Tablica}}
\subsubsection[{zmianarozmiaru}]{\setlength{\rightskip}{0pt plus 5cm}void Tablica\-::zmianarozmiaru (
\begin{DoxyParamCaption}
\item[{unsigned int}]{nowyrozmiar}
\end{DoxyParamCaption}
)}}\label{class_tablica_a4e743bdbb74647717d63015894dfae8d}
Funkcja pomocnicza sluzaca do zmiany rozmiaru .

Argumenty i najwazniejsze pola funkcji \begin{DoxyReturn}{Zwraca}
nowyrozmiar -\/\-Rozmiar tablicy po zmianie. 
\end{DoxyReturn}


Definicja w linii 34 pliku tablica.\-cpp.



Dokumentacja dla tej klasy została wygenerowana z plików\-:\begin{DoxyCompactItemize}
\item 
/home/karolina/\-Pulpit/pamsi/prj/inc/\hyperlink{tablica_8hh}{tablica.\-hh}\item 
/home/karolina/\-Pulpit/pamsi/prj/src/\hyperlink{tablica_8cpp}{tablica.\-cpp}\end{DoxyCompactItemize}

\chapter{Dokumentacja plików}
\hypertarget{czas_8hh}{\section{Dokumentacja pliku czas.\-hh}
\label{czas_8hh}\index{czas.\-hh@{czas.\-hh}}
}
{\ttfamily \#include $<$ctime$>$}\\*
{\ttfamily \#include \char`\"{}stos.\-hh\char`\"{}}\\*
{\ttfamily \#include \char`\"{}stos2.\-hh\char`\"{}}\\*
{\ttfamily \#include \char`\"{}stoslista.\-hh\char`\"{}}\\*
{\ttfamily \#include \char`\"{}kolejka.\-hh\char`\"{}}\\*
Wykres zależności załączania dla czas.\-hh\-:
Ten wykres pokazuje, które pliki bezpośrednio lub pośrednio załączają ten plik\-:
\subsection*{Komponenty}
\begin{DoxyCompactItemize}
\item 
class \hyperlink{class_czasstos}{Czasstos$<$ klasasto $>$}
\begin{DoxyCompactList}\small\item\em Sablon klasy \hyperlink{class_czasstos}{Czasstos}. Klasa \hyperlink{class_czasstos}{Czasstos} mierzy czas dzialania algorytmu wykorzystuje $\ast$parametry takie jak \-: liczba powtorzen i zmienna-\/czyli rozmiar problemu . \end{DoxyCompactList}\item 
class \hyperlink{class_czaskol}{Czaskol$<$ klasakol $>$}
\end{DoxyCompactItemize}

\hypertarget{dzialania_8hh}{\section{Dokumentacja pliku /home/karolina/\-Pulpit/pamsi/prj/inc/dzialania.hh}
\label{dzialania_8hh}\index{/home/karolina/\-Pulpit/pamsi/prj/inc/dzialania.\-hh@{/home/karolina/\-Pulpit/pamsi/prj/inc/dzialania.\-hh}}
}
{\ttfamily \#include \char`\"{}tablica.\-hh\char`\"{}}\\*
Wykres zależności załączania dla dzialania.\-hh\-:
Ten wykres pokazuje, które pliki bezpośrednio lub pośrednio załączają ten plik\-:
\subsection*{Komponenty}
\begin{DoxyCompactItemize}
\item 
class \hyperlink{class_dzialania}{Dzialania}
\begin{DoxyCompactList}\small\item\em Deklaracja klasy \hyperlink{class_dzialania}{Dzialania}. \end{DoxyCompactList}\end{DoxyCompactItemize}

\hypertarget{kolejka_8hh}{\section{Dokumentacja pliku /home/karolina/\-Pulpit/pamsi/prj/inc/kolejka.hh}
\label{kolejka_8hh}\index{/home/karolina/\-Pulpit/pamsi/prj/inc/kolejka.\-hh@{/home/karolina/\-Pulpit/pamsi/prj/inc/kolejka.\-hh}}
}
{\ttfamily \#include $<$list$>$}\\*
Wykres zależności załączania dla kolejka.\-hh\-:\nopagebreak
\begin{figure}[H]
\begin{center}
\leavevmode
\includegraphics[width=202pt]{kolejka_8hh__incl}
\end{center}
\end{figure}
Ten wykres pokazuje, które pliki bezpośrednio lub pośrednio załączają ten plik\-:
\nopagebreak
\begin{figure}[H]
\begin{center}
\leavevmode
\includegraphics[width=202pt]{kolejka_8hh__dep__incl}
\end{center}
\end{figure}
\subsection*{Komponenty}
\begin{DoxyCompactItemize}
\item 
class \hyperlink{class_kolejka}{Kolejka$<$ Typ $>$}
\begin{DoxyCompactList}\small\item\em Szablon klasy \hyperlink{class_kolejka}{Kolejka} Klasa zaimplementowana na liscie. \end{DoxyCompactList}\end{DoxyCompactItemize}

\hypertarget{stos_8hh}{\section{Dokumentacja pliku stos.\-hh}
\label{stos_8hh}\index{stos.\-hh@{stos.\-hh}}
}
{\ttfamily \#include $<$stack$>$}\\*
{\ttfamily \#include \char`\"{}tablica.\-hh\char`\"{}}\\*
Wykres zależności załączania dla stos.\-hh\-:
Ten wykres pokazuje, które pliki bezpośrednio lub pośrednio załączają ten plik\-:
\subsection*{Komponenty}
\begin{DoxyCompactItemize}
\item 
class \hyperlink{class_stos}{Stos$<$ Typ $>$}
\begin{DoxyCompactList}\small\item\em Szablon klasy stos. Klasa stos wykonuja dzialania \-: -\/odklada elementy na stos -\/sciaga element ze stosu -\/zwraca rozmiar -\/sprawdza czy stos jesyt pusty. \end{DoxyCompactList}\end{DoxyCompactItemize}

\hypertarget{stos2_8hh}{\section{Dokumentacja pliku /home/karolina/\-Pulpit/pamsi/prj/inc/stos2.hh}
\label{stos2_8hh}\index{/home/karolina/\-Pulpit/pamsi/prj/inc/stos2.\-hh@{/home/karolina/\-Pulpit/pamsi/prj/inc/stos2.\-hh}}
}
{\ttfamily \#include $<$stack$>$}\\*
{\ttfamily \#include \char`\"{}tablica.\-hh\char`\"{}}\\*
Wykres zależności załączania dla stos2.\-hh\-:\nopagebreak
\begin{figure}[H]
\begin{center}
\leavevmode
\includegraphics[width=264pt]{stos2_8hh__incl}
\end{center}
\end{figure}
Ten wykres pokazuje, które pliki bezpośrednio lub pośrednio załączają ten plik\-:\nopagebreak
\begin{figure}[H]
\begin{center}
\leavevmode
\includegraphics[width=198pt]{stos2_8hh__dep__incl}
\end{center}
\end{figure}
\subsection*{Komponenty}
\begin{DoxyCompactItemize}
\item 
class \hyperlink{class_stos2}{Stos2$<$ Typ $>$}
\begin{DoxyCompactList}\small\item\em Szablon klasy \hyperlink{class_stos2}{Stos2} \hyperlink{class_stos}{Stos} ktory w przypadku przepelnienia zmienia swoj rozmiar. Metody analogiczne do klasy \hyperlink{class_stos}{Stos}. \end{DoxyCompactList}\end{DoxyCompactItemize}

\hypertarget{stoslista_8hh}{\section{Dokumentacja pliku stoslista.\-hh}
\label{stoslista_8hh}\index{stoslista.\-hh@{stoslista.\-hh}}
}
{\ttfamily \#include $<$list$>$}\\*
Wykres zależności załączania dla stoslista.\-hh\-:
Ten wykres pokazuje, które pliki bezpośrednio lub pośrednio załączają ten plik\-:
\subsection*{Komponenty}
\begin{DoxyCompactItemize}
\item 
class \hyperlink{class_stol}{Stol$<$ Typ $>$}
\begin{DoxyCompactList}\small\item\em Szablon klasy \hyperlink{class_stol}{Stol}. \end{DoxyCompactList}\end{DoxyCompactItemize}

\hypertarget{tablica_8hh}{\section{Dokumentacja pliku tablica.\-hh}
\label{tablica_8hh}\index{tablica.\-hh@{tablica.\-hh}}
}
{\ttfamily \#include $<$iostream$>$}\\*
{\ttfamily \#include $<$cstdlib$>$}\\*
Wykres zależności załączania dla tablica.\-hh\-:
Ten wykres pokazuje, które pliki bezpośrednio lub pośrednio załączają ten plik\-:
\subsection*{Komponenty}
\begin{DoxyCompactItemize}
\item 
class \hyperlink{class_tablica}{Tablica$<$ Typ $>$}
\begin{DoxyCompactList}\small\item\em Szablon klasy \hyperlink{class_tablica}{Tablica}. \end{DoxyCompactList}\end{DoxyCompactItemize}

\hypertarget{dzialania_8cpp}{\section{Dokumentacja pliku /home/karolina/\-Pulpit/pamsi/prj/src/dzialania.cpp}
\label{dzialania_8cpp}\index{/home/karolina/\-Pulpit/pamsi/prj/src/dzialania.\-cpp@{/home/karolina/\-Pulpit/pamsi/prj/src/dzialania.\-cpp}}
}
{\ttfamily \#include $<$iostream$>$}\\*
{\ttfamily \#include $<$fstream$>$}\\*
{\ttfamily \#include $<$dzialania.\-hh$>$}\\*
{\ttfamily \#include $<$ctime$>$}\\*
Wykres zależności załączania dla dzialania.\-cpp\-:

\hypertarget{main_8cpp}{\section{Dokumentacja pliku /home/karolina/\-Pulpit/pamsi/prj/src/main.cpp}
\label{main_8cpp}\index{/home/karolina/\-Pulpit/pamsi/prj/src/main.\-cpp@{/home/karolina/\-Pulpit/pamsi/prj/src/main.\-cpp}}
}
{\ttfamily \#include $<$iostream$>$}\\*
{\ttfamily \#include $<$dzialania.\-hh$>$}\\*
{\ttfamily \#include $<$cstdlib$>$}\\*
Wykres zależności załączania dla main.\-cpp\-:
\subsection*{Funkcje}
\begin{DoxyCompactItemize}
\item 
int \hyperlink{main_8cpp_a3c04138a5bfe5d72780bb7e82a18e627}{main} (int argc, char $\ast$$\ast$argv)
\begin{DoxyCompactList}\small\item\em Funkcja main wykonuje algorytm i sprawdza czas dzialania algorytmu. W funkcji main wykonywane sa operacje \-: -\/\-Wczytywanie pliku z danymi wejsciowym -\/\-Dane sa mnozone razy 2 (w tej chwili wlaczany jest stoper) -\/\-Wczytywanie pliku z danymi sprawdzajacymi -\/\-Sprawdzanie zgodnosci -\/\-Stoper zostaje wylaczony i na wyjsciu programu podany zostaje czas dzialania algorytmu. \end{DoxyCompactList}\end{DoxyCompactItemize}


\subsection{Dokumentacja funkcji}
\hypertarget{main_8cpp_a3c04138a5bfe5d72780bb7e82a18e627}{\index{main.\-cpp@{main.\-cpp}!main@{main}}
\index{main@{main}!main.cpp@{main.\-cpp}}
\subsubsection[{main}]{\setlength{\rightskip}{0pt plus 5cm}int main (
\begin{DoxyParamCaption}
\item[{int}]{argc, }
\item[{char $\ast$$\ast$}]{argv}
\end{DoxyParamCaption}
)}}\label{main_8cpp_a3c04138a5bfe5d72780bb7e82a18e627}


Funkcja main wykonuje algorytm i sprawdza czas dzialania algorytmu. W funkcji main wykonywane sa operacje \-: -\/\-Wczytywanie pliku z danymi wejsciowym -\/\-Dane sa mnozone razy 2 (w tej chwili wlaczany jest stoper) -\/\-Wczytywanie pliku z danymi sprawdzajacymi -\/\-Sprawdzanie zgodnosci -\/\-Stoper zostaje wylaczony i na wyjsciu programu podany zostaje czas dzialania algorytmu. 



Definicja w linii 22 pliku main.\-cpp.


\hypertarget{tablica_8cpp}{\section{Dokumentacja pliku tablica.\-cpp}
\label{tablica_8cpp}\index{tablica.\-cpp@{tablica.\-cpp}}
}
{\ttfamily \#include $<$tablica.\-hh$>$}\\*
{\ttfamily \#include $<$iostream$>$}\\*
Wykres zależności załączania dla tablica.\-cpp\-:

\addcontentsline{toc}{part}{Indeks}
\printindex
\end{document}
