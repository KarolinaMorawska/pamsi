
\documentclass[11pt]{article} % use larger type; default would be 10pt
\usepackage{amsmath}
\usepackage[polish]{babel}
\usepackage[utf8]{inputenc}
\usepackage{polski}
\usepackage{url}
\frenchspacing
\usepackage{graphicx}

%%% PAGE DIMENSIONS
\usepackage{geometry} 
\geometry{a4paper} 
\usepackage{graphicx} 

%%% PACKAGES
\usepackage{booktabs} % for much better looking tables
\usepackage{array} % for better arrays (eg matrices) in maths
\usepackage{paralist} % very flexible & customisable lists (eg. enumerate/itemize, etc.)
\usepackage{verbatim} % adds environment for commenting out blocks of text & for better verbatim
\usepackage{subfig} % make it possible to include more than one captioned figure/table in a single float
% These packages are all incorporated in the memoir class to one degree or another...

%%% HEADERS & FOOTERS
\usepackage{fancyhdr} % This should be set AFTER setting up the page geometry
\pagestyle{fancy} % options: empty , plain , fancy
\renewcommand{\headrulewidth}{0pt} % customise the layout...
\lhead{}\chead{}\rhead{}
\lfoot{}\cfoot{\thepage}\rfoot{}

%%% SECTION TITLE APPEARANCE
\usepackage{sectsty}
\allsectionsfont{\sffamily\mdseries\upshape} 
\usepackage[nottoc,notlof,notlot]{tocbibind} % Put the bibliography in the ToC
\usepackage[titles,subfigure]{tocloft} % Alter the style of the Table of Contents
\renewcommand{\cftsecfont}{\rmfamily\mdseries\upshape}
\renewcommand{\cftsecpagefont}{\rmfamily\mdseries\upshape} % No bold!

%%% END Article customizations

%%% The "real" document content comes below...

\title{Sprawozdanie z ćwiczenia laboratoryjnego \\,,Grafy''}
\author{Karolina Morawska}
%\date{} % Activate to display a given date or no date (if empty),
         % otherwise the current date is printed 

\begin{document}
\maketitle
\newpage
\section{Czym wogóle jest graf ?}
 To taka struktura danych, która składa się z wierzchołków i krawedzi, przy czym poszczególne wierzchołki (zwane też węzłami) mogą być połączone krawędziami (skierowanymi lub nieskierowanymi) w taki sposób, iż każda krawędź zaczyna się i kończy w którymś z wierzchołków. Wierzchołki i krawędzie mogą być numerowane, etykietowane i nieść pewną dodatkową informację - w zależności od potrzeby modelu, do którego konstrukcji są wykorzystane. W porównaniu do drzew w grafach mogą występować pętle i cykle. Krawędzie mogą mieć wyznaczony kierunek (wtedy graf nazywamy skierowanym), mogą mieć przypisaną wagę (pewną liczbę), kolor, etykietę, np. odległość pomiędzy punktami w terenie, rodzaj połączenia.


Istnieje wiele rodzajów grafów, które mogą mieć wiele interesujących właściwości. Grafy mogą być, np.:

\begin{itemize}
\item Skierowane gdy możliwe jest przejście pomiędzy wierzchołkami tylko w jedną stronę (krawędź wtedy oznaczamy strzałką).
\item Nieskierowane gdy możliwe jest przejście pomiędzy wierzchołkami w obydwie strony.
\end{itemize}
Naszym zadaniem było zaimplementowanie grafu nieskierowanego.
\subsection{Opis implementacji grafu nieskierowanego wybranego przeze mnie.}

Wybrałam implememtacje grafu za pomocą listy sąsiedztwa.Reprezentacja grafu za pomocą list sąsiedztwa jest podobna do reprezentacji macierzą sąsiedztwa. Mamy tablicę n-elementową, gdzie n oznacza liczbę wierzchołków w grafie. Każdy element tej tablicy jest skojarzony z jednym wierzchołkiem grafu - numer wiersza jest numerem wierzchołka. Elementy tablicy są listami. Listy te zawierają numery wierzchołków w grafie, do których prowadzi z danego wierzchołka krawędź.
\\Zaletą takiej implementacji jest:
\begin{itemize}
\item  Oszczędność pamięci komputera, ponieważ odwzorywane są  tylko istniejące krawędzie.
\item Dostęp do sąsiadów danego wierzchołka jest szybszy niż w przypadku tablicy sąsiedztwa, ponieważ nie musimy sprawdzać kolejnych wierzchołków - lista od razu zawiera gotowych do odczytu sąsiadów.
\end{itemize}
\newpage
\section{Algorytmu służące do przeszukiwania grafu}
\subsection{Breadth-first search, czyli przeszukiwanie wszerz}
Jest to jeden z najprostszych algorytmów przeszukiwań służacy do odnajdywania najkrótszej drogi w grafie. Przechodzenie grafu rozpoczyna się od zadanego wierzchołka i polega na odwiedzeniu wszystkich dostępnych z niego wierzchołków.

Złożoność pamięciowa algorytmu uzależniona jest od sposobu implementacji grafu.W moim przypadku czyli implementacji grafu za pomocą listy sąsiedztwa dla każdego wierzchołka przechowywana jest lista wierzchołków dostępnych bezpośrednio z niego.Złożoność pamięciowa wynosi $O(|V|+|E|)$ gdzie$|V|$ to liczba węzłów a $|E|$ to liczba krawędzi w grafie.

Ponieważ w najgorszym przypadku przeszukiwanie wszerz musi przebyć wszystkie krawędzie prowadzące do wszystkich węzłów, złożoność czasowa tego przeszukiwania wynosi$ O(|V| + |E|)$, gdzie $|V|$ to liczba węzłów, a $|E|$ to liczba krawędzi w grafie.

\subsection{Depth-first search, czyli przeszukiwanie w głąb}
Polega na badaniu wszystkich krawędzi wychodzących z podanego wierzchołka.Po zbadaniu wszystkich krawędzi wychodzących danego wierzchołka algorytm powraca do wierzchołka, z którego dany wierzchołek został odwiedzony.

Złożoność pamięciowa algorytmu w przypadku drzewa jest o wiele mniejsza niż przeszukiwania wszerz, gdyż algorytm w każdym momencie wymaga zapamiętania tylko ścieżki od korzenia do bieżącego węzła, podczas gdy przeszukiwanie wszerz wymaga zapamiętywania wszystkich węzłów w danej odległości od korzenia, co zwykle rośnie wykładniczo w funkcji długości ścieżki.

Złożoność czasowa przeszukiwania jest uzależniona od liczby wierzchołków oraz liczby krawędzi. Algorytm musi odwiedzić wszystkie wierzchołki oraz wszystkie krawędzie, co oznacza, że złożoność wynosi $O(|V|+|E|)$.
\newpage
\section{Wyszukiwanie ścieżek w grafie}
\subsection{ ,,Algorytm Dijkstry''.}


Aby wyznaczyć najkrótszą ścieżkę między s a wszystkimi innymi węzłami, w algorytmie Dijkstry wymaga się, aby w każdym węźle zapisywane były kolor i oszacowanie najkrótszej ścieżki. Początkowo wszystkim węzłom przypisujemy kolor biały, wszystkie oszacowania ścieżki ustawiamy na $\infty$ . Oszacowanie najkrótszej ścieżki dla węzła początkowego ustawiamy na 0.
W miare działania algorytmu, wszystkim węzłom poza poczatkowym przypisujemy rodziców z drzewa najkrótszych ścieżek. Rodzic węzła może zmieniać się przed zakończeniem działania algorytmu wielokrotnie.
Dalej algorytm działa nastepująco: najpierw sposród wszystkich białych węzłów grafu wybieramy węzeł u z najmniejszym oszacowaniem najkrótszej ścieżki. Wstępnie będzie to węzeł początkowy, którego ścieżka została oszacowana na 0. Po wybraniu węzła zaczerniamy go. Następnie, dla każdego białego węzła v przylegającego do u zwalniamy krawędź (u, v). Kiedy zwalniamy krawędź, sprawdzamy, czy przejście z u do v poprawi wyznaczoną dotąd najkrótszą ścieżkę do v. W tym celu dodajemy wagę (u, v) do oszacowania najkrótszej ścieżki do u. Jeśli wartość ta jest mniejsza lub równa oszacowaniu najkrótszej ścieżki do v, przypisujemy tę wartość v jako nowe oszacowanie najkrótszej scieżki i ustawiamy v jako rodzica u. Proces ten powtarzamy dotąd, aż wszystkie węzły będa czarne. Kiedy wyliczone zostanie już drzewo najkrótszych ścieżek, najkrótszą ścieżkę z węzła s do danego węzła t można wybrać poprzez przejście po tym drzewie od węzła t przez kolejnych rodziców, aż do s. Ścieżka o odwrotnej kolejności do uzyskanej jest ścieżką szukaną. 

\subsection{,,Algorytm Forda-Bellmana''}
Algorytm ten służy do wyznaczania najmniejszej odległości od ustalonego wierzchołka s do wszystkich pozostałych w skierowanym grafie bez cykli.
Warunek braku cykli jest spowodowany faktem, że w grafie posiadajacym cykl najmniejsza odległość między niektórymi wierzchołkami jest nieokreślona, ponieważ zależy od liczby przejść w cyklu.
 Macierz A dla każdej pary wierzchołków u i v zawiera wagę krawędzi (u,v), przy czym jeśli krawędź (u,v) nie istnieje, to przyjmujemy, że jej waga wynosi nieskończoność. Algorytm Forda-Bellmana w każdym kroku oblicza górne oszacowanie S(vi) odległości od wierzchołka s do wszystkich pozostałych wierzchołków vi. W pierwszym kroku przyjmujemy S(vi)=A(s,vi). Gdy stwierdzimy, że S(v)$>$S(u)+A(u,v), to każdorazowo polepszamy aktualne oszacowanie i podstawiamy S(v):=S(u)+A(u,v). Algorytm kończy się, gdy żadnego oszacowania nie można już poprawić, macierz S(vi) zawiera najkrótsze odległości od wierzchołka s do wszystkich pozostałych.
 \newpage
\begin{thebibliography}{99}
\bibitem{pa}
\emph{$ http://pl.wikipedia.org/wiki/Algorytm_Dijkstry$,}
\bibitem{pa}
\emph{$http://pl.wikipedia.org/wiki/Algorytm_Bellmana-Forda$}
\bibitem{pa}
\emph{$http://zasoby1.open.agh.edu.pl/dydaktyka/informatyka$}

\bibitem{pa}
\emph{$http://pl.wikipedia.org/wiki/Przeszukiwanie_wszerz$}

\bibitem{pa}
\emph{$http://pl.wikipedia.org/wiki/Przeszukiwanie_w$}

\bibitem{pa}
\emph{$http://www.algorytm.org/algorytmy$-$grafowe/przeszukiwanie$-$grafu$-$wszerz$-$bfs$-$i$-$w$-$glab$-$dfs.html$}
\end{thebibliography}
\end{document}
