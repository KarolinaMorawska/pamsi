\documentclass[11pt]{article} 
\usepackage[T1]{fontenc}
 \usepackage[polish]{babel} \usepackage[utf8]{inputenc}
 \usepackage{lmodern} 
\usepackage{graphicx} 
\selectlanguage{polish} 
\begin{document}
 \begin{titlepage} 
\title{ Sprawozdanie z laboratorium nr5 -PAMSI} 
\author{Karolina Morawska}
 \date{30 03 2014}
 \maketitle 
\end{titlepage}
 \underline{Zadanie do wykonania}
\\Ocena złożoności obliczeniowej poszczególnych algorytmów sortowania . 
\begin{center}


 \begin{figure}[ht!] 
\centering
 \includegraphics[width=90mm]{intro1.jpg}
 \caption{Wykres zależności rozmiaru problemu od czasu dzialania algorytmu dla sortowania introspektywnego.  } 
\label{overflow}
 \end{figure}


 \begin{figure}[ht!] 
\centering
 \includegraphics[width=90mm]{intro1.jpg}
 \caption{Wykres zależności rozmiaru problemu od czasu dzialania algorytmu dla sortowania introspektywnego.} 
\label{overflow}
 \end{figure}

 \begin{figure}[h!] 
\centering
 \includegraphics[width=90mm]{intro2.jpg}
 \caption{Wykres zależności rozmiaru problemu od czasu dzialania algorytmu dla sortowania introspektywnego.}
\label{overflow}
 \end{figure}

 \begin{figure}[h!] 
\centering
 \includegraphics[width=90mm]{intro3.jpg}
 \caption{Wykres zależności rozmiaru problemu od czasu dzialania algorytmu dla sortowania introspektywnego.}
\label{overflow}
 \end{figure}

 \begin{figure}[h!] 
\centering
 \includegraphics[width=90mm]{merge.jpg}
 \caption{Wykres zależności rozmiaru problemu od czasu dzialania algorytmu dla sortowania merge czyli przez scalanie} 
\label{overflow}
 \end{figure}

 \begin{figure}[h] 
\centering
 \includegraphics[width=90mm]{szybkie.jpg}
 \caption{Wykres zależności rozmiaru problemu od czasu dzialania algorytmu dla sortowania szybkiego - 3 różne przypadki } 
\label{overflow}
 \end{figure}

 \end{center}
\newpage 
 Wybrano sortowanie szybkie którego złożoność obliczeniowa wynosi $O(n log n)$.\\Jest to sortowanie wydajne oraz często używane ze względu na prostote implementacji i szybkość. 
Następnym sortowaniem  jest mergesort-rekurencyjny algorytm sortowania danych który dziala w taki oto  sposob:
   \\- dzieli zestaw danych na dwie równe części
  \\-  stosuje sortowanie przez scalanie dla każdej z nich oddzielnie, chyba że pozostał już tylko jeden element;
   \\ -łaczy posortowane podciągi w jeden.
\\Jego złożoność  obliczeniowa jest taka sama jak w przypadku quicksort.

Ostatnie sortowanie to introspektywne.W przypadku ogólnym, a więc  również w najgorszym, algorytm Sortowania Introspektywnego posiada złożonosc obliczeniową taką samą jak 2  poprzednie.
W najgorszym przypadku algorytm wykonuje najpierw rekurencyjne wywołanie, takie jak w Sortowaniu Szybkim, a następnie dla pozostałego podzbioru wywołuje procedurę Heap Sort. Jest on algorytmem sortujacym w miejscu.
\newpage 
Wnioski:
\\Czas sortowania dla zbiorów nieuporządkowanych jest dużo większy od czasu sortowania dla zbiorów uporządkowanych.
\\Optymistyczny przypadek sortowania zachodzi tylko i wyłącznie wtedy gdy kluczem podziału jest mediana z sortowanego zbioru.Podział daje nam równe zbiory.
\\ W przypadku przeciętnym , gdzie rozkład prawdopodobieństwa wyboru elementu jest taki sam złożoność jest zaledwie wyższa o 39% niż w przypadku optymistycznym.
\\ Przypadek pesemistyczny zachodzi wtedy, gdy każdy podział jest skrajnie nirównomierny. tj. generuje jedno podzadanie puste,a drugie o1 mniejsze niz całe zadanie .
\\Wszystkie otrzymane czasy sortowania są proporcjonalne do iloczynu n log2 n, możemy zauważyć zatem , iż klasa złożoności obliczeniowej algorytmu sortowania szybkiego jest równa $O(n log n)$.


 \end{document}