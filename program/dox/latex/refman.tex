\documentclass{book}
\usepackage[a4paper,top=2.5cm,bottom=2.5cm,left=2.5cm,right=2.5cm]{geometry}
\usepackage{makeidx}
\usepackage{natbib}
\usepackage{graphicx}
\usepackage{multicol}
\usepackage{float}
\usepackage{listings}
\usepackage{color}
\usepackage{ifthen}
\usepackage[table]{xcolor}
\usepackage{textcomp}
\usepackage{alltt}
\usepackage{ifpdf}
\ifpdf
\usepackage[pdftex,
            pagebackref=true,
            colorlinks=true,
            linkcolor=blue,
            unicode
           ]{hyperref}
\else
\usepackage[ps2pdf,
            pagebackref=true,
            colorlinks=true,
            linkcolor=blue,
            unicode
           ]{hyperref}
\usepackage{pspicture}
\fi
\usepackage[utf8]{inputenc}
\usepackage{polski}
\usepackage[T1]{fontenc}

\usepackage{mathptmx}
\usepackage[scaled=.90]{helvet}
\usepackage{courier}
\usepackage{sectsty}
\usepackage{amssymb}
\usepackage[titles]{tocloft}
\usepackage{doxygen}
\lstset{language=C++,inputencoding=utf8,basicstyle=\footnotesize,breaklines=true,breakatwhitespace=true,tabsize=4,numbers=left }
\makeindex
\setcounter{tocdepth}{3}
\renewcommand{\footrulewidth}{0.4pt}
\renewcommand{\familydefault}{\sfdefault}
\hfuzz=15pt
\setlength{\emergencystretch}{15pt}
\hbadness=750
\tolerance=750
\begin{document}
\hypersetup{pageanchor=false,citecolor=blue}
\begin{titlepage}
\vspace*{7cm}
\begin{center}
{\Large My Project }\\
\vspace*{1cm}
{\large Wygenerowano przez Doxygen 1.8.3.1}\\
\vspace*{0.5cm}
{\small Śr, 23 kwi 2014 23:25:07}\\
\end{center}
\end{titlepage}
\clearemptydoublepage
\pagenumbering{roman}
\tableofcontents
\clearemptydoublepage
\pagenumbering{arabic}
\hypersetup{pageanchor=true,citecolor=blue}
\chapter{Dokumentacja zadania lab6}
\label{index}\hypertarget{index}{}\begin{DoxyDate}{Data}
\-: 30 kwi 2014 
\end{DoxyDate}
\begin{DoxyAuthor}{Autor}
\-: Karolina Morawska 
\end{DoxyAuthor}
\begin{DoxyVersion}{Wersja}
\-: 0.\-1 
\end{DoxyVersion}

\chapter{Indeks klas}
\section{Lista klas}
Tutaj znajdują się klasy, struktury, unie i interfejsy wraz z ich krótkimi opisami\-:\begin{DoxyCompactList}
\item\contentsline{section}{\hyperlink{class_graf}{Graf} \\*Modeluje pojęcie \hyperlink{class_graf}{Graf}. Klasa modeluje pojęcie graf . Jej atrybutem jest pole zawierajace liste sasedztwa }{\pageref{class_graf}}{}
\item\contentsline{section}{\hyperlink{classpara}{para} \\*Modeluje pojęcie para. Klasa modeluje pojęcie para . Jej atrybutem sa pola zawierajace numer wierzcholka i jego wage }{\pageref{classpara}}{}
\end{DoxyCompactList}

\chapter{Indeks plików}
\section{Lista plików}
Tutaj znajduje się lista wszystkich plików z ich krótkimi opisami\-:\begin{DoxyCompactList}
\item\contentsline{section}{\hyperlink{dzialania_8cpp}{dzialania.\-cpp} \\*Metod klasy \hyperlink{class_dzialania}{Dzialania} }{\pageref{dzialania_8cpp}}{}
\item\contentsline{section}{\hyperlink{dzialania_8hh}{dzialania.\-hh} \\*Definicja klasy \hyperlink{class_dzialania}{Dzialania} }{\pageref{dzialania_8hh}}{}
\item\contentsline{section}{\hyperlink{main_8cpp}{main.\-cpp} \\*Glowny plik programu }{\pageref{main_8cpp}}{}
\item\contentsline{section}{\hyperlink{tablica_8cpp}{tablica.\-cpp} \\*Metody klasy \hyperlink{class_tablica}{Tablica} }{\pageref{tablica_8cpp}}{}
\item\contentsline{section}{\hyperlink{tablica_8hh}{tablica.\-hh} \\*Definicja klasy \hyperlink{class_tablica}{Tablica} }{\pageref{tablica_8hh}}{}
\end{DoxyCompactList}

\chapter{Dokumentacja klas}
\hypertarget{class_n_o_d_e}{\section{Dokumentacja klasy N\-O\-D\-E}
\label{class_n_o_d_e}\index{N\-O\-D\-E@{N\-O\-D\-E}}
}


Modeluje pojęcie \hyperlink{class_n_o_d_e}{N\-O\-D\-E} ,czyli wezel. Klasa modeluje pojęcie node . Jej atrybutem są pola zawierające wskaźnik na lewy ,prawy węzeł i wartość.  




{\ttfamily \#include $<$node.\-hh$>$}



Diagram współpracy dla N\-O\-D\-E\-:\nopagebreak
\begin{figure}[H]
\begin{center}
\leavevmode
\includegraphics[width=166pt]{class_n_o_d_e__coll__graph}
\end{center}
\end{figure}
\subsection*{Metody publiczne}
\begin{DoxyCompactItemize}
\item 
\hyperlink{class_n_o_d_e_ad0f54f9f701169888e42f58a2eb8d0b5}{N\-O\-D\-E} ()
\end{DoxyCompactItemize}
\subsection*{Atrybuty publiczne}
\begin{DoxyCompactItemize}
\item 
\hyperlink{class_n_o_d_e}{N\-O\-D\-E} $\ast$ \hyperlink{class_n_o_d_e_a79a9fdf3352d7eeab59e13ee550ec744}{left}
\item 
\hyperlink{class_n_o_d_e}{N\-O\-D\-E} $\ast$ \hyperlink{class_n_o_d_e_a569656ce6663a21e6f00aa3de1128c90}{right}
\item 
\hyperlink{class_para}{Para} \hyperlink{class_n_o_d_e_a1a2031f75d5d983b3e1aef6913a6cc4b}{value}
\end{DoxyCompactItemize}


\subsection{Opis szczegółowy}


Definicja w linii 26 pliku node.\-hh.



\subsection{Dokumentacja konstruktora i destruktora}
\hypertarget{class_n_o_d_e_ad0f54f9f701169888e42f58a2eb8d0b5}{\index{N\-O\-D\-E@{N\-O\-D\-E}!N\-O\-D\-E@{N\-O\-D\-E}}
\index{N\-O\-D\-E@{N\-O\-D\-E}!NODE@{N\-O\-D\-E}}
\subsubsection[{N\-O\-D\-E}]{\setlength{\rightskip}{0pt plus 5cm}N\-O\-D\-E\-::\-N\-O\-D\-E (
\begin{DoxyParamCaption}
{}
\end{DoxyParamCaption}
)\hspace{0.3cm}{\ttfamily [inline]}}}\label{class_n_o_d_e_ad0f54f9f701169888e42f58a2eb8d0b5}


Definicja w linii 44 pliku node.\-hh.



\subsection{Dokumentacja atrybutów składowych}
\hypertarget{class_n_o_d_e_a79a9fdf3352d7eeab59e13ee550ec744}{\index{N\-O\-D\-E@{N\-O\-D\-E}!left@{left}}
\index{left@{left}!NODE@{N\-O\-D\-E}}
\subsubsection[{left}]{\setlength{\rightskip}{0pt plus 5cm}{\bf N\-O\-D\-E}$\ast$ N\-O\-D\-E\-::left}}\label{class_n_o_d_e_a79a9fdf3352d7eeab59e13ee550ec744}


Definicja w linii 31 pliku node.\-hh.

\hypertarget{class_n_o_d_e_a569656ce6663a21e6f00aa3de1128c90}{\index{N\-O\-D\-E@{N\-O\-D\-E}!right@{right}}
\index{right@{right}!NODE@{N\-O\-D\-E}}
\subsubsection[{right}]{\setlength{\rightskip}{0pt plus 5cm}{\bf N\-O\-D\-E}$\ast$ N\-O\-D\-E\-::right}}\label{class_n_o_d_e_a569656ce6663a21e6f00aa3de1128c90}


Definicja w linii 35 pliku node.\-hh.

\hypertarget{class_n_o_d_e_a1a2031f75d5d983b3e1aef6913a6cc4b}{\index{N\-O\-D\-E@{N\-O\-D\-E}!value@{value}}
\index{value@{value}!NODE@{N\-O\-D\-E}}
\subsubsection[{value}]{\setlength{\rightskip}{0pt plus 5cm}{\bf Para} N\-O\-D\-E\-::value}}\label{class_n_o_d_e_a1a2031f75d5d983b3e1aef6913a6cc4b}


Definicja w linii 39 pliku node.\-hh.



Dokumentacja dla tej klasy została wygenerowana z pliku\-:\begin{DoxyCompactItemize}
\item 
/home/karolina/\-Pulpit/pamsi6/prj/inc/\hyperlink{node_8hh}{node.\-hh}\end{DoxyCompactItemize}

\hypertarget{class_para}{\section{Dokumentacja klasy Para}
\label{class_para}\index{Para@{Para}}
}


Modeluje pojęcie \hyperlink{class_para}{Para}. Klasa modeluje pojęcie para . Jej atrybutem są pola zawierające klucz i wartość.  




{\ttfamily \#include $<$para.\-hh$>$}

\subsection*{Metody publiczne}
\begin{DoxyCompactItemize}
\item 
\hyperlink{class_para_a391b57d370246309feb169f708b33b8f}{Para} ()
\begin{DoxyCompactList}\small\item\em Konstruktor bezparametryczny. \end{DoxyCompactList}\item 
\hyperlink{class_para_aa167d3206ddb657c90a11144c52354d8}{Para} (int \-\_\-klucz, int \-\_\-wartosc)
\end{DoxyCompactItemize}
\subsection*{Atrybuty publiczne}
\begin{DoxyCompactItemize}
\item 
int \hyperlink{class_para_a4defe3990664c62fba0c75f3b2fc0f61}{klucz}
\begin{DoxyCompactList}\small\item\em Inicjalizuje wartosc klucza. \end{DoxyCompactList}\item 
int \hyperlink{class_para_a0c99183281c295298b3fb99f8795a591}{wartosc}
\begin{DoxyCompactList}\small\item\em Inicjalizuje wartosc wartosci. \end{DoxyCompactList}\end{DoxyCompactItemize}


\subsection{Opis szczegółowy}


Definicja w linii 24 pliku para.\-hh.



\subsection{Dokumentacja konstruktora i destruktora}
\hypertarget{class_para_a391b57d370246309feb169f708b33b8f}{\index{Para@{Para}!Para@{Para}}
\index{Para@{Para}!Para@{Para}}
\subsubsection[{Para}]{\setlength{\rightskip}{0pt plus 5cm}Para\-::\-Para (
\begin{DoxyParamCaption}
{}
\end{DoxyParamCaption}
)\hspace{0.3cm}{\ttfamily [inline]}}}\label{class_para_a391b57d370246309feb169f708b33b8f}


Definicja w linii 37 pliku para.\-hh.

\hypertarget{class_para_aa167d3206ddb657c90a11144c52354d8}{\index{Para@{Para}!Para@{Para}}
\index{Para@{Para}!Para@{Para}}
\subsubsection[{Para}]{\setlength{\rightskip}{0pt plus 5cm}Para\-::\-Para (
\begin{DoxyParamCaption}
\item[{int}]{\-\_\-klucz, }
\item[{int}]{\-\_\-wartosc}
\end{DoxyParamCaption}
)\hspace{0.3cm}{\ttfamily [inline]}}}\label{class_para_aa167d3206ddb657c90a11144c52354d8}


Definicja w linii 41 pliku para.\-hh.



\subsection{Dokumentacja atrybutów składowych}
\hypertarget{class_para_a4defe3990664c62fba0c75f3b2fc0f61}{\index{Para@{Para}!klucz@{klucz}}
\index{klucz@{klucz}!Para@{Para}}
\subsubsection[{klucz}]{\setlength{\rightskip}{0pt plus 5cm}int Para\-::klucz}}\label{class_para_a4defe3990664c62fba0c75f3b2fc0f61}


Definicja w linii 29 pliku para.\-hh.

\hypertarget{class_para_a0c99183281c295298b3fb99f8795a591}{\index{Para@{Para}!wartosc@{wartosc}}
\index{wartosc@{wartosc}!Para@{Para}}
\subsubsection[{wartosc}]{\setlength{\rightskip}{0pt plus 5cm}int Para\-::wartosc}}\label{class_para_a0c99183281c295298b3fb99f8795a591}


Definicja w linii 33 pliku para.\-hh.



Dokumentacja dla tej klasy została wygenerowana z pliku\-:\begin{DoxyCompactItemize}
\item 
/home/karolina/\-Pulpit/pamsi6/prj/inc/\hyperlink{para_8hh}{para.\-hh}\end{DoxyCompactItemize}

\input{classper}
\input{class_tablicaas}
\hypertarget{class_tree}{\section{Dokumentacja klasy Tree}
\label{class_tree}\index{Tree@{Tree}}
}


Modeluje pojęcie \hyperlink{class_para}{Para}. Klasa modeluje pojęcie para . Jej atrybutem jest pole zawierajace root(korzen)  




{\ttfamily \#include $<$tree.\-hh$>$}



Diagram współpracy dla Tree\-:\nopagebreak
\begin{figure}[H]
\begin{center}
\leavevmode
\includegraphics[width=166pt]{class_tree__coll__graph}
\end{center}
\end{figure}
\subsection*{Metody publiczne}
\begin{DoxyCompactItemize}
\item 
void \hyperlink{class_tree_ab7fedd8767bedee15602ac3611a85d8a}{dodaj} (\hyperlink{class_para}{Para} element)
\item 
int \hyperlink{class_tree_a1289bf12ad417b94c64b37c340ee4d1e}{wyszukaj} (int key)
\item 
void \hyperlink{class_tree_af8902c205c23f79e8ad7befec9f4cd3c}{usun} (int key)
\item 
\hyperlink{class_tree_ad376a7c639d857312f5de2ef47482f68}{Tree} ()
\end{DoxyCompactItemize}
\subsection*{Atrybuty publiczne}
\begin{DoxyCompactItemize}
\item 
\hyperlink{class_n_o_d_e}{N\-O\-D\-E} $\ast$ \hyperlink{class_tree_a9b7c5c70165244116357f68463c19f0b}{root}
\end{DoxyCompactItemize}


\subsection{Opis szczegółowy}


Definicja w linii 27 pliku tree.\-hh.



\subsection{Dokumentacja konstruktora i destruktora}
\hypertarget{class_tree_ad376a7c639d857312f5de2ef47482f68}{\index{Tree@{Tree}!Tree@{Tree}}
\index{Tree@{Tree}!Tree@{Tree}}
\subsubsection[{Tree}]{\setlength{\rightskip}{0pt plus 5cm}Tree\-::\-Tree (
\begin{DoxyParamCaption}
{}
\end{DoxyParamCaption}
)\hspace{0.3cm}{\ttfamily [inline]}}}\label{class_tree_ad376a7c639d857312f5de2ef47482f68}


Definicja w linii 49 pliku tree.\-hh.



\subsection{Dokumentacja funkcji składowych}
\hypertarget{class_tree_ab7fedd8767bedee15602ac3611a85d8a}{\index{Tree@{Tree}!dodaj@{dodaj}}
\index{dodaj@{dodaj}!Tree@{Tree}}
\subsubsection[{dodaj}]{\setlength{\rightskip}{0pt plus 5cm}void Tree\-::dodaj (
\begin{DoxyParamCaption}
\item[{{\bf Para}}]{element}
\end{DoxyParamCaption}
)}}\label{class_tree_ab7fedd8767bedee15602ac3611a85d8a}


Definicja w linii 56 pliku tree.\-hh.

\hypertarget{class_tree_af8902c205c23f79e8ad7befec9f4cd3c}{\index{Tree@{Tree}!usun@{usun}}
\index{usun@{usun}!Tree@{Tree}}
\subsubsection[{usun}]{\setlength{\rightskip}{0pt plus 5cm}void Tree\-::usun (
\begin{DoxyParamCaption}
\item[{int}]{key}
\end{DoxyParamCaption}
)}}\label{class_tree_af8902c205c23f79e8ad7befec9f4cd3c}


Definicja w linii 106 pliku tree.\-hh.

\hypertarget{class_tree_a1289bf12ad417b94c64b37c340ee4d1e}{\index{Tree@{Tree}!wyszukaj@{wyszukaj}}
\index{wyszukaj@{wyszukaj}!Tree@{Tree}}
\subsubsection[{wyszukaj}]{\setlength{\rightskip}{0pt plus 5cm}int Tree\-::wyszukaj (
\begin{DoxyParamCaption}
\item[{int}]{key}
\end{DoxyParamCaption}
)}}\label{class_tree_a1289bf12ad417b94c64b37c340ee4d1e}


Definicja w linii 91 pliku tree.\-hh.



\subsection{Dokumentacja atrybutów składowych}
\hypertarget{class_tree_a9b7c5c70165244116357f68463c19f0b}{\index{Tree@{Tree}!root@{root}}
\index{root@{root}!Tree@{Tree}}
\subsubsection[{root}]{\setlength{\rightskip}{0pt plus 5cm}{\bf N\-O\-D\-E}$\ast$ Tree\-::root}}\label{class_tree_a9b7c5c70165244116357f68463c19f0b}


Definicja w linii 32 pliku tree.\-hh.



Dokumentacja dla tej klasy została wygenerowana z pliku\-:\begin{DoxyCompactItemize}
\item 
/home/karolina/\-Pulpit/pamsi6/prj/inc/\hyperlink{tree_8hh}{tree.\-hh}\end{DoxyCompactItemize}

\chapter{Dokumentacja plików}
\hypertarget{node_8hh}{\section{Dokumentacja pliku /home/karolina/\-Pulpit/pamsi6/prj/inc/node.hh}
\label{node_8hh}\index{/home/karolina/\-Pulpit/pamsi6/prj/inc/node.\-hh@{/home/karolina/\-Pulpit/pamsi6/prj/inc/node.\-hh}}
}


Definicja klasy N\-O\-D\-E(wezel)  


{\ttfamily \#include \char`\"{}para.\-hh\char`\"{}}\\*
{\ttfamily \#include $<$cstdlib$>$}\\*
Wykres zależności załączania dla node.\-hh\-:\nopagebreak
\begin{figure}[H]
\begin{center}
\leavevmode
\includegraphics[width=198pt]{node_8hh__incl}
\end{center}
\end{figure}
Ten wykres pokazuje, które pliki bezpośrednio lub pośrednio załączają ten plik\-:\nopagebreak
\begin{figure}[H]
\begin{center}
\leavevmode
\includegraphics[width=204pt]{node_8hh__dep__incl}
\end{center}
\end{figure}
\subsection*{Komponenty}
\begin{DoxyCompactItemize}
\item 
class \hyperlink{class_n_o_d_e}{N\-O\-D\-E}
\begin{DoxyCompactList}\small\item\em Modeluje pojęcie \hyperlink{class_n_o_d_e}{N\-O\-D\-E} ,czyli wezel. Klasa modeluje pojęcie node . Jej atrybutem są pola zawierające wskaźnik na lewy ,prawy węzeł i wartość. \end{DoxyCompactList}\end{DoxyCompactItemize}


\subsection{Opis szczegółowy}
Plik zawiera definicję klasy \hyperlink{class_n_o_d_e}{N\-O\-D\-E} ,która jest klasą podrzędną i jest ona specjalizacją klasy \hyperlink{class_tree}{Tree}. 

Definicja w pliku \hyperlink{node_8hh_source}{node.\-hh}.


\hypertarget{para_8hh}{\section{Dokumentacja pliku /home/karolina/\-Pulpit/pamsi6/prj/inc/para.hh}
\label{para_8hh}\index{/home/karolina/\-Pulpit/pamsi6/prj/inc/para.\-hh@{/home/karolina/\-Pulpit/pamsi6/prj/inc/para.\-hh}}
}


Definicja klasy \hyperlink{class_para}{Para}.  


Ten wykres pokazuje, które pliki bezpośrednio lub pośrednio załączają ten plik\-:\nopagebreak
\begin{figure}[H]
\begin{center}
\leavevmode
\includegraphics[width=204pt]{para_8hh__dep__incl}
\end{center}
\end{figure}
\subsection*{Komponenty}
\begin{DoxyCompactItemize}
\item 
class \hyperlink{class_para}{Para}
\begin{DoxyCompactList}\small\item\em Modeluje pojęcie \hyperlink{class_para}{Para}. Klasa modeluje pojęcie para . Jej atrybutem są pola zawierające klucz i wartość. \end{DoxyCompactList}\end{DoxyCompactItemize}


\subsection{Opis szczegółowy}
Plik zawiera definicję klasy \hyperlink{class_para}{Para} ,która jest klasą podrzędną i jest ona specjalizacją klasy \hyperlink{class_n_o_d_e}{N\-O\-D\-E}. 

Definicja w pliku \hyperlink{para_8hh_source}{para.\-hh}.


\input{per_8hh}
\input{tablicaas_8hh}
\hypertarget{tree_8hh}{\section{Dokumentacja pliku /home/karolina/\-Pulpit/pamsi6/prj/inc/tree.hh}
\label{tree_8hh}\index{/home/karolina/\-Pulpit/pamsi6/prj/inc/tree.\-hh@{/home/karolina/\-Pulpit/pamsi6/prj/inc/tree.\-hh}}
}


Definicja klasy \hyperlink{class_tree}{Tree}.  


{\ttfamily \#include \char`\"{}node.\-hh\char`\"{}}\\*
{\ttfamily \#include $<$iostream$>$}\\*
{\ttfamily \#include $<$cstdlib$>$}\\*
Wykres zależności załączania dla tree.\-hh\-:\nopagebreak
\begin{figure}[H]
\begin{center}
\leavevmode
\includegraphics[width=250pt]{tree_8hh__incl}
\end{center}
\end{figure}
Ten wykres pokazuje, które pliki bezpośrednio lub pośrednio załączają ten plik\-:\nopagebreak
\begin{figure}[H]
\begin{center}
\leavevmode
\includegraphics[width=204pt]{tree_8hh__dep__incl}
\end{center}
\end{figure}
\subsection*{Komponenty}
\begin{DoxyCompactItemize}
\item 
class \hyperlink{class_tree}{Tree}
\begin{DoxyCompactList}\small\item\em Modeluje pojęcie \hyperlink{class_para}{Para}. Klasa modeluje pojęcie para . Jej atrybutem jest pole zawierajace root(korzen) \end{DoxyCompactList}\end{DoxyCompactItemize}


\subsection{Opis szczegółowy}
Plik zawiera definicję klasy \hyperlink{class_tree}{Tree} Jest to klasa glowna , ktora wykorzystuje klase \hyperlink{class_n_o_d_e}{N\-O\-D\-E}. 

Definicja w pliku \hyperlink{tree_8hh_source}{tree.\-hh}.


\hypertarget{czas_8cpp}{\section{Dokumentacja pliku /home/karolina/\-Pulpit/pamsi6/prj/src/czas.cpp}
\label{czas_8cpp}\index{/home/karolina/\-Pulpit/pamsi6/prj/src/czas.\-cpp@{/home/karolina/\-Pulpit/pamsi6/prj/src/czas.\-cpp}}
}


Definicja metody Start.  


{\ttfamily \#include \char`\"{}czas.\-hh\char`\"{}}\\*
Wykres zależności załączania dla czas.\-cpp\-:\nopagebreak
\begin{figure}[H]
\begin{center}
\leavevmode
\includegraphics[width=204pt]{czas_8cpp__incl}
\end{center}
\end{figure}


\subsection{Opis szczegółowy}
Definicja metody Wynik.

Definicja metody Koniec.

Metoda, ktora wlacza zegar.

Metoda, ktora powoduje zatrzymanie zegara i liczy czas dzialania algorytmu.

Metoda, ktora wyswietla wynik. 

Definicja w pliku \hyperlink{czas_8cpp_source}{czas.\-cpp}.


\hypertarget{main_8cpp}{\section{Dokumentacja pliku /home/karolina/\-Pulpit/pamsi/prj/src/main.cpp}
\label{main_8cpp}\index{/home/karolina/\-Pulpit/pamsi/prj/src/main.\-cpp@{/home/karolina/\-Pulpit/pamsi/prj/src/main.\-cpp}}
}
{\ttfamily \#include $<$iostream$>$}\\*
{\ttfamily \#include $<$dzialania.\-hh$>$}\\*
{\ttfamily \#include $<$cstdlib$>$}\\*
Wykres zależności załączania dla main.\-cpp\-:
\subsection*{Funkcje}
\begin{DoxyCompactItemize}
\item 
int \hyperlink{main_8cpp_a3c04138a5bfe5d72780bb7e82a18e627}{main} (int argc, char $\ast$$\ast$argv)
\begin{DoxyCompactList}\small\item\em Funkcja main wykonuje algorytm i sprawdza czas dzialania algorytmu. W funkcji main wykonywane sa operacje \-: -\/\-Wczytywanie pliku z danymi wejsciowym -\/\-Dane sa mnozone razy 2 (w tej chwili wlaczany jest stoper) -\/\-Wczytywanie pliku z danymi sprawdzajacymi -\/\-Sprawdzanie zgodnosci -\/\-Stoper zostaje wylaczony i na wyjsciu programu podany zostaje czas dzialania algorytmu. \end{DoxyCompactList}\end{DoxyCompactItemize}


\subsection{Dokumentacja funkcji}
\hypertarget{main_8cpp_a3c04138a5bfe5d72780bb7e82a18e627}{\index{main.\-cpp@{main.\-cpp}!main@{main}}
\index{main@{main}!main.cpp@{main.\-cpp}}
\subsubsection[{main}]{\setlength{\rightskip}{0pt plus 5cm}int main (
\begin{DoxyParamCaption}
\item[{int}]{argc, }
\item[{char $\ast$$\ast$}]{argv}
\end{DoxyParamCaption}
)}}\label{main_8cpp_a3c04138a5bfe5d72780bb7e82a18e627}


Funkcja main wykonuje algorytm i sprawdza czas dzialania algorytmu. W funkcji main wykonywane sa operacje \-: -\/\-Wczytywanie pliku z danymi wejsciowym -\/\-Dane sa mnozone razy 2 (w tej chwili wlaczany jest stoper) -\/\-Wczytywanie pliku z danymi sprawdzajacymi -\/\-Sprawdzanie zgodnosci -\/\-Stoper zostaje wylaczony i na wyjsciu programu podany zostaje czas dzialania algorytmu. 



Definicja w linii 22 pliku main.\-cpp.


\addcontentsline{toc}{part}{Indeks}
\printindex
\end{document}
